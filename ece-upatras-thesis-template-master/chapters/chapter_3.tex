%!TEX root = ../main.tex



\section{Μοντελοποίηση προβλήματος}

\lettrine[findent=2pt]{\fbox{\textbf{Π}}}{ριν} δούμε στην πράξη πώς λειτουργούν οι τεχνικές super resolution θα πρέπει να περιγράψουμε το πρόβλημα με το οποίο θα δουλέψουμε με μαθηματικούς όρους. Η μοντελοποίηση αυτή θα μας επιτρέψει να χρησιμοποιήσουμε μαθηματικά εργαλεία και τεχνικές και να ορίσουμε σε μια αυστηρή ``γλώσσα'' (αυτή των μαθηματικών) τις λειτουργίες που επιτελούνται από κάθε μέθοδο, ούτως ώστε να πετύχουμε το τελικό αποτέλεσμα.

Ξεκινώντας, θα πρέπει να περιγράψουμε τον τρόπο με τον οποίο λαμβάνουμε εικόνες χαμηλής ανάλυσης από μια φυσική σκηνή μέσω μιας κάμερας. Η κάμερα σαν όργανο καταγραφής εισάγει κάποιες ατέλειες, όπως είδαμε στο προηγούμενο κεφάλαιο. Τέτοιες ατέλειες μπορεί να είναι σφάλματα καταγραφής από τον αισθητήρα της κάμερας, θόλωμα λόγω αστοχιών των οπτικών στοιχείων κλπ. Επιπλέον, αν λάβουμε διαδοχικές εικόνες μιας φυσικής σκηνής, οι εικόνες αυτές περιμένουμε να έχουν κάποιες μετατοπίσεις ως προς αυτό που απεικονίζουν. Οι μετατοπίσεις αυτές μπορεί να οφείλονται είτε στον άνθρώπινο παράγοντα (που χειρίζεται την κάμερα) είτε στο αντικείμενο της φυσικής σκηνής που μπορεί να μην είναι σταθερό. Στην εικόνα που λαμβάνεται τελικά, λαμβάνονται δείγματα σε χαμηλή χωρική συχνότητα και λόγω ατελειών του οργάνου μπορεί να έχουμε και παρουσία θορύβου. Για μια εικόνα υψηλής ανάλυσης $\bm{X}$ (την οποία θα προσπαθήσουμε να ανακατασκευάσουμε), μπορούμε να αναπτύξουμε το παραπάνω μοντέλο ως εξής:
\begin{equation} \label{eq:hrmodel}
\bm{Y}_i=\bm{S}_i\bm{T}_i\bm{H}_i\bm{X}+\bm{n}_i
\end{equation}
όπου ορίζουμε για την i-οστή εικόνα χαμηλής ανάλυσης $\bm{Y}_i$ τους τελεστές που ενεργούν στην $\bm{X}$ : 
\begin{itemize}
	\item $\bm{S}_{i}$ για υποδειγματοληψία, 
	\item $\bm{T}_{i}$ για μετατόπιση, 
	\item $\bm{H}_{i}$ για θόλωμα (blurring), 
	\item $\bm{n_{i}}$ για προσθετικό θόρυβο.
\end{itemize}
Οι υποθέσεις που κάνουμε για το παραπάνω πρόβλημα είναι ότι το blurring είναι ίδιο σε όλο το χώρο και είναι γνωστό στον αλγόριθμο super resolution, ο θόρυβος είναι λευκός Gaussian με την ίδια διασπορά σε όλες τις εικόνες χαμηλής ανάλυσης και ότι ο γεωμετρικός μετασχηματισμός αφορά μόνο την καθολική μετατόπιση. 

 Στην περίπτωση που θεωρήσουμε αμελητέο θόρυβο και θόλωμα, τα παραπάνω βήματα αρκούν για να υπολογίσουμε την εικόνα υψηλής ανάλυσης $\bm{X}$, την οποία λαμβάνουμε σαν αποτέλεσμα του αλγορίθμου.

\begin{algorithm}

 \KwData{$\bm{dx}$, $\bm{dy}$, $\bm{Y}_i$, $N$, $W$, $\bm{H}$, $S$}
 \KwOut{High resolution reconstructed $\bm{X}$}
 $\bm{X} \gets 0$\; 
 \For{$n \gets 1$ \textbf{to} $N$}{
 	$\bm{\tilde{Y}} \gets \bm{Y}_i$\;
    $\bm{i} \gets 1:W/S$\;
    $\bm{j} \gets 1:H/S$\;
    $\bm{px}=\bm{i}*S+\bm{dx}_n$\;
    $\bm{py}=\bm{j}*S+\bm{dy}_n$\;
    $\bm{X}_{px,py}=\bm{\tilde{Y}}_{i,j}$\;
    }
    %\vspace{-1.5em}
 \caption{Ανακατασκευή shift-add fusion}\label{algo:sr_fusion}
\end{algorithm}

Συγκρίνοντας τη μέθοδο αυτή με την ανακατασκευή μέσω του μετασχηματισμού Fourier, παρατηρούμε την απλότητα και την αποδοτικότητά της καθώς βρίσκει την εικόνα υψηλής ανάλυσης μόνο με μετακινήσεις pixel. 