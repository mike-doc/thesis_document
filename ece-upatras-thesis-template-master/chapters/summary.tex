\pagestyle{plain}
\begin{center}
{\LARGE Περίληψη}\\[1cm]
\end{center}

%\lettrine[findent=2pt]{\fbox{\textbf{H}}}{ } εργασία αυτή ασχολείται με ένα ιδιαίτερα ενδιαφέρον ζήτημα στο χώρο της επεξεργασίας σημάτων και εικόνων, την ανάλυση (\emph{resolution}). Παρόλο που σήμερα στον κόσμο μας έχουμε καταφέρει να δημιουργήσουμε συσκευές με μεγάλη ευαισθησία καταγραφής, μεγάλο χώρο αποθήκευσης καθώς και υψηλούς ρυθμούς μετάδοσης και επεξεργασίας δεδομένων, εντούτοις υπάρχουν εφαρμογές όπου η φύση τους είναι τέτοια που δε μας επιτρέπει να επωφεληθούμε σε μεγάλο βαθμό από την πρόοδο που έχει σημειωθεί. Μια τέτοια εφαρμογή είναι οι θερμικές εικόνες και θα δούμε στη συνέχεια της εργασίας τους λόγους εκείνους που την καθιστούν ``ιδιαίτερη".

\lettrine[findent=2pt]{\fbox{\textbf{Η}}}{ } εργασία αυτή ασχολείται με την υλοποίηση και την προσομοίωση σε περιβάλλον LabVIEW, ενός αυτο--ρυθμιζόμενου PID ελεγκτή. Παρόλο που οι PID ελεγκτές αποτελούν ένα πολύ διαδεδομένο είδος ελεγκτών με ευρεία χρήση σε βιομηχανικές, και όχι μόνο, εφαρμογές η σωστή ρύθμιση τους απαιτεί εμπειρία από το χειριστή και συνήθως αποτελεί χρονοβόρα διαδικασία.

Πολλοί τρόποι έχουν αναπτυχθεί για την αυτόματη ρύθμιση των κερδών ενός PID ελεγκτή έτσι ώστε ο έλεγχος που προσφέρει να πληρεί κάποιες απαιτήσεις. Στην εργασία αυτή θα ασχοληθούμε με τη μέθοδο \emph{Relay} πάνω στην οποία βασίστηκε ο συγκεκριμένος αυτο--ρυθμιζόμενος PID ελεγκτής που σχεδιάστηκε. 

Χρησιμοποιώντας τη γλώσσα προγραμματισμού LabVIEW αναπτύχθηκε, εκτός από τον αλγόριθμο υπολογισμού των κερδών, και το γραφικό περιβάλλον του ελεγκτή. Αυτό προσφέρει ποικίλες επιλογές, όπως ο χειροκίνητος ορισμός των κερδών του ελεγκτή ή ο αυτόματος υπολογισμός τους, επιλογή της μεθόδου υπολογισμού των κερδών, επιλογή της επιθυμητής ταχύτητας απόκρισης του κλειστού συστήματος και αυτόματη προσαρμογή των κερδών σε νέες συνθήκες εργασίας.

Για την αξιολόγηση του ελεγκτή και του ελέγχου που προσφέρει, έχουν μοντελοποιηθεί επτά κλασικά συστήματα ελέγχου. Το γραφικό περιβάλλον που έχει αναπτυχθεί παρουσιάζει το μοντέλο κάθε συστήματος καθώς και τις μεταβλητές που επηρεάζουν τη δυναμική του. Η προσομοίωση επιτρέπει να δούμε, σε πραγματικό χρόνο, πώς ανταποκρίνεται το σύστημα όταν σε αυτό επιδρούν διάφοροι παράγοντες όπως διαταραχές ή αλλαγή του επιθυμητού σημείου του.