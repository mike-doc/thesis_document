%!TEX root = ../main.tex



\section{Εισαγωγή στους PID ελεγκτές}

\lettrine[findent=2pt]{\fbox{\textbf{Έ}}}{νας} αναλογικός-ολοκληρωτικός-παραγωγικός ελεγκτής (\emph{proportional-integral-derivative controller}) ή όπως είναι πιο γνωστός \emph{PID controller}, είναι ένας μηχανισμός ανάδρασης (\emph{feedback}) βρόχου ελέγχου (\emph{control loop}) που χρησιμοποιείται ευρέως σε βιομηχανικά συστήματα ελέγχου καθώς και σε μια ποικιλία άλλων εφαρμογών που απαιτούν συνεχή διαμορφωμένο έλεγχο. Η διαδικασία λειτουργίας είναι κοινή για όλους τους ελεγκτές αυτού του είδους. Ένας PID ελεγκτής υπολογίζει συνεχώς μια τιμή σφάλματος $e(t)$ ως διαφορά μεταξύ μιας επιθυμητής τιμής ρύθμισης (\emph{setpoint} ή \emph{SP}) και μεταξύ μιας μεταβλητής της διαδικασίας ύπο έλεγχο (\emph{process value} ή \emph{PV}) και εφαρμόζει μια διόρθωση βασισμένη στον αναλογικό, ολοκληρωτικό και παραγωγικό όρο του (\emph{P}, \emph{I}, \emph{D} αντίστοιχα) οι οποίοι δίνουν και στον ελεγκτή το όνομά του.\\
\linebreak
Στην πράξη, εφαρμόζει αυτόματα διορθωμένη και ακριβή διόρθωση σε μια λειτουργία ελέγχου. Ένα καθημερινό παράδειγμα είναι ο έλεγχος ταχύτητας σε οδικό όχημα. όπου εξωτερικές επιδράσεις, όπως κλίσεις, θα προκαλούσαν αλλαγές στην ταχύτητα του οχήματος. Ο αλγόριθμος PID επαναφέρει την ταχύτητα του αυτοκινήτου στην επιθυμητή από τον οδηγό τιμή της με τον βέλτιστο τρόπο, χωρίς καθυστέρηση ή υπέρβαση, ελέγχοντας την ισχύ εξόδου του κινητήρα του οχήματος.

\subsection{Πρώτη εφαρμογή}

Η πρώτη θεωρητική ανάλυση και πρακτική εφαρμογή αφορούσε το πεδίο του αυτόματου συστήματος διεύθυνσης πλοίων, το οποίο αναπτύχθηκε από τις αρχές της δεκαετίας του 1920 και μετά. Στη συνέχεια χρησιμοποιήθηκε για τον αυτόματο έλεγχο της διαδικασίας στη μεταποιητική βιομηχανία, όπου εφαρμόστηκε ευρέως σε πνευματικούς, και στη συνέχεια ηλεκτρονικούς, ελεγκτές. Σήμερα υπάρχει γενική χρήση της ιδέας PID σε εφαρμογές που απαιτούν ακριβή και βελτιστοποιημένο αυτόματο έλεγχο.










%\lettrine[findent=2pt]{\fbox{\textbf{Κ}}}{ατά} την αποτύπωση οποιουδήποτε αναλογικού σήματος $S(t)$ σε ψηφιακή μορφή $S_i$, πραγματοποιείται μια διαδικασία που ονομάζεται \emph{δειγματοληψία (sampling)} κατά την οποία ένα συνεχές σήμα γίνεται διακριτό και στη συνέχεια εφαρμόζουμε \emph{κβαντισμό} στις τιμές του, για να περάσουμε σε ψηφιακή αναπαράσταση. Όπως φαίνεται και στο σχήμα \ref{fig:sampling}, σε διακριτές χρονικές στιγμές λαμβάνουμε την τιμή του αναλογικού σήματος (μέσω ενός μετατροπέα Αναλογικού σε Ψηφιακό (ADC)) κι έτσι δημιουργούμε μια σειρά από τιμές (\emph{δείγματα}) τα οποία έχουν συγκεκριμένη και σταθερή χρονική απόσταση $T_s$. Η χρονική απόσταση έχει νόημα μόνο ως προς το αναλογικό σήμα, ενώ στην ψηφιακή μορφή των δειγμάτων αναφερόμαστε σε αυτά με ένα δείκτη $n$ και μεταξύ αναλογικού και ψηφιακού σήματος ισχύει ότι $S_i=S(nT_s)$.
%\begin{figure}[h]
%  \centering
%  \includegraphics[width=0.6\textwidth]{Signal_Sampling}
%  \caption{Δειγματοληψία ενός αναλογικού σήματος}
%  \label{fig:sampling}
%\end{figure}
%Κατά τη δειγματοληψία ``χάνουμε'' αρκετή από την πληροφορία που περιέχει το αναλογικό σήμα, καθώς οι τιμές που βρίσκονται μεταξύ των διαστημάτων $T_s$ όπου λαμβάνουμε δείγματα δε λαμβάνονται υπόψη. Ακόμη, λόγω της πεπερασμένης ακρίβειας των ψηφιακών συστημάτων για αναπαράσταση αριθμών, η τιμή που έχει το σήμα στρογγυλοποιείται στην πλησιέστερη (είτε μεγαλύτερη είτε μικρότερη) τιμή που μπορεί να αναπαραστήσει το σύστημά μας.
%
%Όπως γίνεται αντιληπτό, σε πολλές των περιπτώσεων η δειγματοληψία μπορεί να έχει καταστροφικές συνέπειες για το αναλογικό σήμα. Ωστόσο, υπό προϋποθέσεις, μπορεί να είναι αντιστρεπτή διαδικασία -- μπορούμε δηλαδή από τα δείγματα που έχουμε λάβει να επιστρέψουμε στην αναλογική μορφή του σήματος. Οι προϋποθέσεις αυτές ορίζονται από το θεώρημα \emph{Shannon-Nyquist} (\ref{thrm:shannon-nyquist}), ως:
%\\
%\begin{theorem}[Shannon-Nyquist]
%	\label{thrm:shannon-nyquist}
%	Ένα σήμα με μέγιστη συχνότητα $f_{max}$ μπορεί να ανακτηθεί από τα δείγματά του, αν αυτά ληφθούν με συχνότητα $f_s>2f_{max}$, ή αλλιώς με περίοδο $T_s<\frac{1}{2f_{max}}$. \cite{proakis_sampling}
%\end{theorem}