%!TEX root = ../main.tex



\section{Εισαγωγή}

\lettrine[findent=2pt]{\fbox{\textbf{Σ}}}{ε} αυτό το κεφάλαιο θα παρουσιαστούν τα πειράματα που έγιναν, έτσι ώστε να υπάρχει εφαρμογή της θεωρίας στην πράξη όσων έχουν αναφερθεί στα προηγούμενα κεφάλαια. Με αυτόν τον τρόπο μελετάται και κατά πόσο ο ελεγκτής που σχεδιάστηκε είναι ικανός να ελέγξει ικανοποιητικά συστήματα ελέγχου με διαφορετικές ανάγκες και ιδιαιτερότητες. Σε κάθε ενότητα θα παρουσιάζεται ένα από τα διαθέσιμα συστήματα, το μαθηματικό μοντέλο του\footnote{Η μοντελοποίηση του κάθε συστήματος βασίζεται στις αναλύσεις που βρίσκονται στο \texttt{\url{http://ctms.engin.umich.edu/CTMS/index.php?aux=Home}}} και τα αποτελέσματα του ελέγχου που παρέχει ο αυτο--ρυθμιζόμενος PID ελεγκτής. Τέλος, η κάθε ενότητα κλείνει με το σχολιασμό των πειραματικών αποτελεσμάτων.

\section{Σύστημα Mass -- Spring -- Damper} \label{sec:mass_spring_damper}

\subsection{Εισαγωγή}

Το πρώτο σύστημα που θα αναλυθεί είναι το σύστημα Μάζα -- Ελατήριο -- Αποσβεστήρας (Σχήμα \ref{fig:mass_spring_damper}) που αποτελεί ένα από τα πιο κλασικά συστήματα ελέγχου καθώς οι δυναμικές σχέσεις που το διέπουν δεν είναι περίπλοκες και είναι εύκολη η κατανόηση τους.

\begin{figure}[h]
  \centering
  \includegraphics[width=\textwidth,height=5cm,keepaspectratio]{mass_spring_damper}
  \caption{Μοντέλο συστήματος Μάζας -- Ελατηρίου -- Αποσβεστήρα}
  \label{fig:mass_spring_damper}
\end{figure}

\subsection{Μαθηματικό Μοντέλο}

\begin{figure}[h]
  \centering
  \includegraphics[width=\textwidth,height=3cm,keepaspectratio]{mass_spring_damper_FBD}
  \caption{Διάγραμμα ελευθέρου σώματος για το σύστημα Μάζα -- Ελατήριο -- Αποσβεστήρας}
  \label{fig:mass_spring_damper_FBD}
\end{figure}

Το διάγραμμα ελευθέρου σώματος για αυτό το σύστημα φαίνεται στο παραπάνω σχήμα. H διαφορική εξίσωση που χαρακτηρίζει το σύστημα αυτό είναι
\begin{equation}
m\ddot{x} + b\dot{x} + kx = F
\label{eq:mass_springer_damper_ode}
\end{equation}
όπου $m$ είναι η μάζα, $x$ είναι η μετατόπιση της μάζας από το σημείο ισορροπίας, $b$ είναι η απόσβεση που παρέχει ο αποσβεστήρας, $k$ είναι η σταθερά του ελατηρίου και $F$ είναι η δύναμη που ασκείται στο σύστημα. Παίρνοντας το μετασχηματισμό Laplace της παραπάνω εξίσωσης έχουμε
\begin{equation}
ms^2X(s) + bsX(s) + kX(s) = F(s)
\label{eq:mass_springer_damper_laplace}
\end{equation}
Συνεπώς η συνάρτηση μεταφοράς (\emph{Transfer Function}) μεταξύ της εισόδου, που είναι η δύναμη $F(s)$, και της εξόδου, που είναι η μετατόπιση της μάζας $X(s)$, είναι
\begin{equation}
\frac{X(s)}{F(s)} = \frac{1}{ms^2 + bs + k} \left[\frac{m}{N}\right]
\end{equation}

\subsection{Πείραμα}
\noindent
Έστω ότι 
\begin{flushleft}
\begin{tabular}{lll}
\textbf{m} & μάζα του σώματος & $1\ kg$ \\ 
\textbf{b} & σταθερά απόσβεσης & $10\ N\frac{s}{m}$ \\ 
\textbf{k} & σταθερά ελατηρίου & $20\ \frac{N}{m}$ \\ 
\end{tabular} 
\end{flushleft}
Αντικαθιστώντας αυτές τις τιμές στην εξίσωση \ref{eq:mass_springer_damper_laplace} έχουμε
\begin{equation}
\frac{X(s)}{F(s)} = \frac{1}{s^2 + 10s + 20}
\end{equation}
Τώρα θέλουμε να δούμε πώς αποκρίνεται το σύστημα σε μια βηματική είσοδο πλάτους $F=1$ Newton. Αρχικά, θα ελέγξουμε τη βηματική του απόκριση (\emph{step response}) όταν δεν εφαρμόζεται έλεγχος.\newline \underline{Σημείωση:} Στα πλαίσια αυτής της εργασίας θεωρείται ότι το σύστημα δρα χωρίς έλεγχο όταν $K_p = 1, T_i = 0$ και $T_d = 0$.

\subsubsection{Απόκριση Χωρίς Έλεγχο}

Στο Σχήμα \ref{fig:mass_springer_damper_no_control} φαίνεται η βηματική απόκριση του συστήματος όταν δεν υπάρχει έλεγχος. Καθίσταται εμφανές ότι από μόνο του το σύστημα έχει μη ικανοποιητική απόκριση καθώς η τελική τιμή της εξόδου του είναι $y(t) = 0,047619$ και έχει ποσοστό σφάλματος \textit{offset error}$= 95.238 \%$. Συνεπώς κρίνεται απαραίτητη η χρήση ελεγκτή. Τα κέρδη του ελεγκτή θα υπολογιστούν χρησιμοποιώντας τη μέθοδο αυτο-ρύθμισης που έχει περιγραφεί.

\begin{figure}[h]
  \centering
  \includegraphics[width=\textwidth,height=5cm,keepaspectratio]{mass_springer_damper_no_control}
  \caption{Βηματική απόκριση του συστήματος Μάζα -- Ελατήριο -- Αποσβεστήρας χωρίς έλεγχο}
  \label{fig:mass_springer_damper_no_control}
\end{figure}

\subsubsection{Αναλογικός Έλεγχος}

Στο Σχήμα \ref{fig:mass_springer_damper_proportional_control} φαίνεται η βηματική απόκριση του συστήματος όταν σε αυτό εφαρμόζεται αναλογικός έλεγχος. Το κέρδος $K_c$ του αναλογικού όρου υπολογίστηκε αυτόματα, χρησιμοποιώντας τους τύπους από τη μέθοδο Ziegler -- Nichols. Βλέπουμε ότι με τη χρήση του αναλογικού ελέγχου το σφάλμα βελτιώθηκε στο βαθμό η τελική τιμή του συστήματος να έχει πλέον μόνο $3,624\%$ απόκλιση από την επιθυμητή τιμή, αλλά, όπως είχε αναφερθεί και στην ενότητα \ref{subsec:proportional_control}, δεν μπορεί να το μηδενίσει. Επίσης, ο έλεγχος εισήγαγε ταλαντώσεις και υπερακόντιση κατά ένα ποσοστό περίπου $50\%$. Αυτό, ανάλογα με τις απαιτήσεις ελέγχου, μπορεί να μην είναι αποδεκτό.

Στο ίδιο σχήμα φαίνεται η αριθμητική τιμή του πλάτους και της περιόδου των ταλαντώσεων που υπολόγισε ο αλγόριθμος. Το Σχήμα \ref{fig:mass_springer_damper_oscillations} αποτελεί μεγέθυνση του \ref{fig:mass_springer_damper_proportional_control} και αποδεικνύει ότι ο αλγόριθμος είναι πολύ ακριβής στην εύρεση των χαρακτηριστικών των ταλαντώσεων. 

\begin{figure}[h]
  \centering
  \includegraphics[width=\textwidth,height=5cm,keepaspectratio]{mass_springer_damper_proportional_control}
  \caption{Βηματική απόκριση του συστήματος Μάζα -- Ελατήριο -- Αποσβεστήρας με εφαρμογή αναλογικού ελέγχου}
  \label{fig:mass_springer_damper_proportional_control}
\end{figure}

\begin{figure}[h]
  \centering
  \includegraphics[width=\textwidth]{mass_springer_damper_oscillations}
  \caption{Μεγέθυνση των ταλαντώσεων του συστήματος κατά τη διάρκεια του πειράματος relay}
  \label{fig:mass_springer_damper_oscillations}
\end{figure}

\subsubsection{Αναλογικός -- Ολοκληρωτικός Έλεγχος}

Το πείραμα επαναλαμβάνεται αλλά αυτή τη φορά χρησιμοποιείται και ο ολοκληρωτικός όρος προκειμένου να εξαλειφθεί το σφάλμα μόνιμης κατάστασης. Όπως φαίνεται από το Σχήμα \ref{fig:mass_springer_damper_integral_control}, η προσθήκη του ολοκληρωτικού όρου, όχι μόνο δεν μηδενίζει το σφάλμα μόνιμης κατάστασης αλλά χειροτερεύει την απόκριση του συστήματος σε σημείο να το οδηγεί να εκτελεί ταλαντώσεις γύρω από το επιθυμητό σημείο. Αυτό χωρίς αμφιβολία, είναι μία μη αποδεκτή κατάσταση.

\begin{figure}[h]
  \centering
  \includegraphics[width=\textwidth,height=5cm,keepaspectratio]{mass_springer_damper_integral_control}
  \caption{Βηματική απόκριση του συστήματος Μάζα -- Ελατήριο -- Αποσβεστήρας με εφαρμογή αναλογικού -- ολοκληρωτικού ελέγχου}
  \label{fig:mass_springer_damper_integral_control}
\end{figure}

\subsubsection{Αναλογικός -- Ολοκληρωτικός -- Διαφορικός Έλεγχος}

Προκειμένου να βελτιωθεί η ευστάθεια του συστήματος εισάγεται και ο διαφορικός όρος. Ενώ ο ολοκληρωτικός όρος εισάγει έναν πόλο στο σύστημα, ο διαφορικός όρος εισάγει ένα μηδενικό βελτιώνοντας έτσι την ευστάθεια. Όπως φαίνεται στο Σχήμα \ref{fig:mass_springer_damper_derivative_control} η απόκριση βελτιώθηκε αισθητά. Ο έλεγχος που παρέχουν και οι τρεις όροι, όχι μόνο μηδένισε το σφάλμα μόνιμης κατάστασης όπως ήταν αναμενόμενο λόγω του ολοκληρωτικού όρου, αλλά βελτίωσε και τη μεταβατική κατάσταση του συστήματος. Η υπερακόντιση, που πριν είχε ποσοστό σχεδόν $50\%$ τώρα έχει ποσοστό περίπου $20\%$. Επίσης έχει χρόνο ανύψωσης λιγότερο από ένα δευτερόλεπτο. Πλέον η απόκριση του συστήματος κρίνεται ικανοποιητική.

\begin{figure}[h]
  \centering
  \includegraphics[width=\textwidth,height=5cm,keepaspectratio]{mass_springer_damper_derivative_control}
  \caption{Βηματική απόκριση του συστήματος Μάζα -- Ελατήριο -- Αποσβεστήρας με εφαρμογή αναλογικού -- ολοκληρωτικού -- διαφορικού ελέγχου}
  \label{fig:mass_springer_damper_derivative_control}
\end{figure}

Όπως έχει αναφερθεί κάθε διεργασία μπορεί να έχει διαφορετικές απαιτήσεις ελέγχου. Μπορεί για μία συγκεκριμένη εφαρμογή, η υπερακόντιση που παρουσιάζει ο έλεγχος με αυτά τα κέρδη να μην είναι αποδεκτός. Έχει ενδιαφέρον λοιπόν να δούμε πώς ανταποκρίνεται το σύστημα σε διαφορετικές τιμές των κερδών. Ορίζοντας την επιθυμητή ταχύτητα του κλειστού συστήματος σε ``Slow" αντί για ``Normal" έχουμε την παρακάτω απόκριση. Από το διάγραμμα βλέπουμε ότι πλέον το σύστημα δεν παρουσιάζει υπερακόντιση αλλά ως αντάλλαγμα αργεί αισθητά περισσότερο να φτάσει στην επιθυμητή τιμή.

\begin{figure}[h]
  \centering
  \includegraphics[width=\textwidth,height=5cm,keepaspectratio]{mass_springer_damper_ZN_slow}
  \caption{Βηματική απόκριση του συστήματος Μάζα -- Ελατήριο -- Αποσβεστήρας με εφαρμογή αναλογικού -- ολοκληρωτικού -- διαφορικού ελέγχου στη λειτουργία ``Slow"}
  \label{fig:mass_springer_damper_ZN_slow}
\end{figure}

\begin{figure}[h]
  \centering
  \includegraphics[width=\textwidth,height=5cm,keepaspectratio]{mass_springer_damper_TL}
  \caption{Βηματική απόκριση του συστήματος Μάζα -- Ελατήριο -- Αποσβεστήρας με εφαρμογή αναλογικού -- ολοκληρωτικού -- διαφορικού ελέγχου του οποίου τα κέρδη έχουν υπολογιστεί με τους τύπους Tyreus -- Luyben}
  \label{fig:mass_springer_damper_TL}
\end{figure}

Ακόμα, στο Σχήμα \ref{fig:mass_springer_damper_TL} φαίνεται η βηματική απόκριση του κλειστού συστήματος όταν τα κέρδη του ελεγκτή έχουν υπολογιστεί με τους τύπους Tyreus -- Luyben. Η απόκριση μοιάζει σαν μια μίξη των δύο προηγούμενων αποκρίσεων. Η έξοδος του συστήματος δεν παρουσιάζει υπερακόντιση, αλλά φτάνει και στην επιθυμητή τιμή μέσα σε περίπου ένα δευτερόλεπτο.

\subsubsection{Αντιμετώπιση Διαταραχών}
Στο Σχήμα \ref{fig:mass_spring_damper_disturbances} φαίνεται πώς το σύστημα αντιδράει στην ύπαρξη διαταραχών. Βλέπουμε ότι το σύστημα αποκρίνεται σχετικά καλά στην ύπαρξη διαταραχών, αφού κρατάει την ελεγχόμενη μεταβλητή κοντά στο επιθυμητό σημείο και την επαναφέρει γρήγορα σε αυτό μετά το πέρας των διαταραχών.

\begin{figure}[h]
  \centering
  \includegraphics[width=\textwidth,height=5cm,keepaspectratio]{mass_spring_damper_disturbances}
  \caption{Αντιμετώπιση διαταραχών του αυτο -- ρυθμιζόμενου PID ελεγκτή του οποίου τα κέρδη έχουν υπολογιστεί με τους τύπους Ziegler -- Nichols}
  \label{fig:mass_spring_damper_disturbances}
\end{figure}

\begin{figure}[h]
  \centering
  \includegraphics[width=\textwidth,height=5cm,keepaspectratio]{mass_spring_damper_disturbances_slow}
  \caption{Αντιμετώπιση διαταραχών του αυτο--ρυθμιζόμενου PID ελεγκτή του οποίου τα κέρδη έχουν υπολογιστεί με τους τύπους Ziegler -- Nichols στη λειτουργία ``Slow"}
  \label{fig:mass_spring_damper_disturbances_slow}
\end{figure}

Όπως και πριν, έχει ενδιαφέρον να δούμε πώς διαχειρίζεται τις διαταραχές το σύστημα όταν τα κέρδη του ελεγκτή έχουν υπολογιστεί χρησιμοποιώντας διαφορετικούς τύπους. Έτσι, στο Σχήμα \ref{fig:mass_spring_damper_disturbances_slow} φαίνεται η προσπάθεια του ελεγκτή να ελέγξει το σύστημα σε διαταραχή, όταν τα κέρδη έχουν υπολογιστεί με τη μέθοδο Ziegler -- Nichols και οι τύποι έχουν προσαρμοστεί για να δίνουν μια πιο αργή απόκριση.

Τέλος, στο Σχήμα \ref{fig:mass_spring_damper_disturbances_TL} φαίνεται η αντιμετώπιση των διαταραχών όταν τα κέρδη έχουν υπολογιστεί με τους τύπους Tyreus -- Luyben. Βλέπουμε ότι οι αντιμετώπιση διαταραχών είναι παρόμοια και στις τρεις περιπτώσεις που εξετάστηκαν.

\begin{figure}[h]
  \centering
  \includegraphics[width=\textwidth,height=5cm,keepaspectratio]{mass_spring_damper_disturbances_TL}
  \caption{Αντιμετώπιση διαταραχών του αυτο--ρυθμιζόμενου PID ελεγκτή του οποίου τα κέρδη έχουν υπολογιστεί με τους τύπους Tyreus -- Luyben}
  \label{fig:mass_spring_damper_disturbances_TL}
\end{figure}

\subsection{Συμπεράσματα}

Στην ενότητα αυτή έγινε προσπάθεια να ελεγχθεί το κλασικό σύστημα ελέγχου που αποτελείται από μία μάζα, ένα ελατήριο και έναν αποσβεστήρα. Ως είσοδος του συστήματος θεωρήθηκε η δύναμη $F$ που ασκείται στη μάζα, ενώ ως έξοδος του συστήματος θεωρήθηκε η μετατόπιση $x$ που έχει η μάζα από το σημείο ισορροπίας της. 

Αρχικά, είδαμε ότι χωρίς έλεγχο το σύστημα δεν ανταποκρίνεται καλά καθώς έχει μεγάλο χρόνο ανύψωσης και μεγάλο σφάλμα μόνιμης κατάστασης. Έτσι στην αρχή χρησιμοποιήθηκε αναλογικός έλεγχος για τη μείωση του σφάλματος και τη βελτίωση του χρόνου ανύψωσης ο οποίος οδήγησε σε μία αξιοπρεπή απόκριση αλλά δεν κατάφερε να μηδενίσει το σφάλμα, παρά μόνο να το μειώσει σε ένα μικρό ποσοστό. 

Στη συνέχεια, χρησιμοποιήθηκε και ο ολοκληρωτικός όρος προκειμένου να μηδενιστεί το σφάλμα μόνιμης κατάστασης. Η τιμή του κέρδους όμως που υπολογίστηκε για τον ολοκληρωτικό όρο είχε σαν αποτέλεσμα να οδηγήσει το σύστημα σε μία μορφή αστάθειας. Αυτό οφείλεται στο ότι τόσο ο αναλογικός όσο και ο ολοκληρωτικός όρος μειώνουν το χρόνο ανύψωσης και ενισχύουν την υπερακόντιση του συστήματος οπότε προσθετικά οδηγούν την έξοδο του συστήματος σε ταλάντωση. Κανονικά, αν θα θέλαμε να έχουμε μόνο αναλογικό -- ολοκληρωτικό έλεγχο θα έπρεπε, πριν την εφαρμογή του ολοκληρωτικού όρου, να μειώσουμε το αναλογικό κέρδος $K_c$. Όμως, στόχος μας στην παρούσα εργασία είναι να δούμε πόσο αποτελεσματικός είναι ο έλεγχος που παρέχει ο αυτο--ρυθμιζόμενος PID ελεγκτής που υλοποιήθηκε, χωρίς να επέμβουμε χειροκίνητα στα κέρδη που υπολογίζει.

Συνεπώς, για να βελτιώσουμε την απόκριση, προστέθηκε και ο διαφορικός όρος του ελεγκτή. Πλέον, ο έλεγχος που περιελάμβανε και τους τρεις όρους οδήγησε το σύστημα στο να έχει γρήγορο χρόνο ανύψωσης, μικρό ποσοστό υπερακόντισης και μηδενικό σφάλμα μόνιμης κατάστασης. Επίσης, αλλάζοντας την ταχύτητα που θέλουμε να έχει το κλειστό σύστημα ελέγχου ή επιλέγοντας τους τύπους Tyreus -- Luyben αντί για τους Ziegler -- Nichols, πετυχαίνουμε μια λιγότερο ``επιθετική" απόκριση. Παρόλο που είναι δύσκολο να πούμε ότι κάποιος έλεγχος είναι ξεκάθαρα καλύτερος από κάποιον άλλο αφού αυτό εξαρτάται από τις απαιτήσεις της κάθε διεργασίας, είναι ασφαλές να δηλώσουμε ότι τα κέρδη που υπολογίζονται με τους τύπους Tyreus -- Luyben συνδυάζουν, ως ένα βαθμό, τις προηγούμενες αποκρίσεις, προσφέροντας έτσι τον πιο ``σφαιρικό" έλεγχο.

\section{Σύστημα Cruise Control} \label{sec:cruise_control}

\subsection{Εισαγωγή}

Το δεύτερο σύστημα που θα αναλυθεί είναι ένα πολύ καλό παράδειγμα ελέγχου ανάδρασης. Ο αυτόματος έλεγχος ταχύτητας ή \emph{cruise control} όπως είναι γνωστό στη διεθνή βιβλιογραφία, είναι ένα χαρακτηριστικό που βρίσκεται σε πολλά σύγχρονα αυτοκίνητα. Στόχος του είναι να κρατάει σταθερή την ταχύτητα του οχήματος, ανεξαρτήτως των διαταραχών που μπορεί να δέχεται από το περιβάλλον του, όπως η αντίσταση του ανέμου ή η κλίση του δρόμου. Αυτό το επιτυγχάνει μετρώντας την τρέχουσα ταχύτητα του οχήματος, συγκρίνοντας την με την επιθυμητή ταχύτητα και αυτόματα προσαρμόζοντας το γκάζι.

\begin{figure}[h]
  \centering
  \includegraphics[width=\textwidth,height=5cm,keepaspectratio]{cruise_control_schematic}
  \caption{Μοντέλο συστήματος Αυτόματου Ελέγχου Ταχύτητας}
  \label{fig:cruise_control_schematic}
\end{figure}

\subsection{Μαθηματικό Μοντέλο}

Για την εύρεση των εξισώσεων του συστήματος, θεωρούμε ένα απλό μοντέλο της δυναμικής του οχήματος που φαίνεται στο παραπάνω διάγραμμα ελευθέρου σώματος. Το όχημα, μάζας $m$, ελέγχεται από μια δύναμη ελέγχου $u$. Η δύναμη αυτή αντιπροσωπεύει τη δύναμη που παράγεται στη διεπαφή δρόμου  --  ελαστικού. Για αυτό το απλοποιημένο μοντέλο, θα υποθέσουμε ότι μπορούμε να ελέγξουμε άμεσα αυτή τη δύναμη και θα παραμελήσουμε τη δυναμική του κινητήρα, των ελαστικών καθώς και των απωλειών που πηγαίνουν στη δημιουργία της δύναμης. Οι δυνάμεις αντίστασης, που οφείλονται στις τριβές των ελαστικών με το δρόμο και την αντίσταση του ανέμου, θεωρούμε ότι μεταβάλλονται γραμμικά με την ταχύτητα του οχήματος, $v$, και ενεργούν προς την αντίθετη κατεύθυνση από αυτή που κινείται το όχημα.

Με αυτές τις υποθέσεις, οδηγούμαστε σε ένα απλό σύστημα πρώτης τάξης. Αθροίζοντας τις δυνάμεις στον άξονα $x$ και εφαρμόζοντας τον $2^o$ Νόμο του Νεύτωνα, καταλήγουμε στην παρακάτω εξίσωση
\begin{equation}
m\dot{\upsilon} + b\upsilon = u
\label{eq:cruise_control_ode}
\end{equation}
όπου $u$ είναι η δύναμη που εφαρμόζεται στο όχημα και αποτελεί την είσοδο του συστήματος και $\upsilon$ είναι η ταχύτητα του οχήματος και αποτελεί την έξοδό του.

Παίρνοντας λοιπόν το μετασχηματισμό Laplace και στα δύο μέλη της εξίσωσης \ref{eq:cruise_control_ode} και θεωρώντας μηδενικές αρχικές συνθήκες, καταλήγουμε στη συνάρτηση μεταφοράς του συστήματος
\begin{equation}
P(s) = \frac{V(s)}{U(s)} = \frac{1}{ms+b} \left[\frac{m/s}{N}\right]
\label{eq:cruise_control_laplace}
\end{equation}

\subsection{Πείραμα}
\noindent
Θεωρούμε ότι οι τιμές των παραμέτρων είναι
\begin{flushleft}
\begin{tabular}{lll}
\textbf{m} & μάζα του συστήματος & $1000\ kg$ \\
\textbf{b} & σταθερά απόσβεσης & $50\ \frac{Ns}{m}$
\end{tabular}
\end{flushleft}
Με αντικατάσταση των τιμών στη συνάρτηση μεταφοράς έχουμε
\begin{equation}
\frac{V(s)}{U(s)} = \frac{1}{1000s+50}
\end{equation}

\subsubsection{Απόκριση Χωρίς Έλεγχο}

Αρχικά, ελέγχουμε να δούμε κατά πόσο το σύστημα χρειάζεται έλεγχο για να έχει ικανοποιητική απόδοση. Η απόκριση του συστήματος όταν ως επιθυμητή ταχύτητα θέτουμε τα $5$ m/s φαίνεται στο παρακάτω σχήμα.

\begin{figure}[h]
  \centering
  \includegraphics[width=\textwidth,height=5cm,keepaspectratio]{cruise_control_no_control}
  \caption{Βηματική απόκριση του συστήματος Αυτόματου Ελέγχου Ταχύτητας χωρίς την εφαρμογή ελέγχου}
  \label{fig:cruise_control_no_control}
\end{figure}

Από την γραφική παράσταση βλέπουμε ότι η απόκριση του συστήματος χωρίς έλεγχο είναι πολύ αργή και έχει πολύ μεγάλο σφάλμα μόνιμης κατάστασης. 

%\subsubsection{Ταλαντώσεις του συστήματος}
%
%Επειδή το συγκεκριμένο σύστημα είναι πρώτης τάξεως, οι ταλαντώσεις που εκτελεί κατά το πείραμα relay, έχουν πολύ μικρό πλάτος και μικρή περίοδο. Το Σχήμα \ref{fig:cruise_control_oscillations} είναι η μεγέθυνση της απόκρισης του συστήματος έτσι ώστε να φανεί το πλάτος και η περίοδος των ταλαντώσεων αυτών. 
%
%\begin{figure}[h]
%  \centering
%  \includegraphics[width=\textwidth,height=5cm,keepaspectratio]{cruise_control_oscillations}
%  \caption{Ταλαντώσεις του συστήματος Αυτομάτου Ελέγχου Ταχύτητας}
%  \label{fig:cruise_control_oscillations}
%\end{figure}

\subsubsection{Αναλογικός Έλεγχος}

Στο Σχήμα \ref{fig:cruise_control_proportional} φαίνεται η απόκριση του συστήματος όταν ο PID ελεγκτής χρησιμοποιεί μόνο τον αναλογικό του όρο. Βλέπουμε ότι για το κέρδος του αναλογικού όρου έχει υπολογιστεί μια υπερβολικά μεγάλη τιμή, συγκεκριμένα $K_c = 176463$. Πέρα από αυτό, η απόκριση του συστήματος πλέον έχει βελτιωθεί καθώς φτάνει στο επιθυμητό σημείο σε περίπου έξι δευτερόλεπτα και το σφάλμα μόνιμης κατάστασης έχει πλέον τιμή $0.028\%$.

\begin{figure}[h]
  \centering
  \includegraphics[width=\textwidth,height=5cm,keepaspectratio]{cruise_control_proportional}
  \caption{Βηματική απόκριση του συστήματος Αυτομάτου Ελέγχου Ταχύτητας με χρήση μόνο του αναλογικού όρου}
  \label{fig:cruise_control_proportional}
\end{figure}

\subsubsection{Αναλογικός  --  Ολοκληρωτικός Έλεγχος}

Στο Σχήμα \ref{fig:cruise_control_integral} φαίνεται η βηματική απόκριση του συστήματος όταν στον έλεγχο προστεθεί και ο ολοκληρωτικός όρος. Τόσο αυτή η απόκριση όσο και η προηγούμενη παρουσιάζουν πολλές ομοιότητες. Ο χρόνος ανύψωσης είναι σχεδόν ίδιος και η έξοδος του συστήματος έχει γραμμική μεταβολή συναρτήσει του χρόνου. Το μόνο που διαφέρει σε σχέση με πριν είναι ότι πλέον, λόγω του ολοκληρωτικού όρου, το σφάλμα μόνιμης κατάστασης είναι εντελώς μηδέν.

\begin{figure}[h]
  \centering
  \includegraphics[width=\textwidth,height=5cm,keepaspectratio]{cruise_control_integral}
  \caption{Βηματική απόκριση του συστήματος Αυτομάτου Ελέγχου Ταχύτητας με χρήση του αναλογικού και του ολοκληρωτικού όρου}
  \label{fig:cruise_control_integral}
\end{figure}

\subsubsection{Αναλογικός  --  Ολοκληρωτικός  --  Διαφορικός Έλεγχος}

Στο Σχήμα \ref{fig:cruise_control_derivative} φαίνεται η απόκριση του συστήματος όταν ο ελεγκτής χρησιμοποιεί και τους τρεις όρους του. Από τη γραφική παράσταση εύκολα βγαίνει το συμπέρασμα ότι η απόκριση δεν άλλαξε καθόλου με την προσθήκη του διαφορικού όρου.

\begin{figure}[h]
  \centering
  \includegraphics[width=\textwidth,height=5cm,keepaspectratio]{cruise_control_derivative}
  \caption{Βηματική απόκριση του συστήματος Αυτομάτου Ελέγχου Ταχύτητας με χρήση και των τριών όρων του PID ελεγκτή}
  \label{fig:cruise_control_derivative}
\end{figure}

Τέλος, στο Σχήμα \ref{fig:cruise_control_TL} φαίνεται η απόκριση του συστήματος όταν για τον αυτόματο υπολογισμό των κερδών του αναλογικού και του ολοκληρωτικού όρου έχουν χρησιμοποιηθεί οι τύποι Tyreus -- Luyben. Όπως ήταν αναμενόμενο, ούτε σε αυτή την περίπτωση η απόκριση του συστήματος παρουσιάζει αλλαγές σε σχέση με τις προηγούμενες γραφικές παραστάσεις.

\begin{figure}[h]
  \centering
  \includegraphics[width=\textwidth,height=5cm,keepaspectratio]{cruise_control_TL}
  \caption{Βηματική απόκριση του συστήματος Αυτομάτου Ελέγχου Ταχύτητας με χρήση και των τριών όρων του PID ελεγκτή όπου τα κέρδη έχουν υπολογιστεί χρησιμοποιώντας τους τύπους Tyreus -- Luyben}
  \label{fig:cruise_control_TL}
\end{figure}

\subsubsection{Αντιμετώπιση Διαταραχών}

Αφού από τα διαδοχικά πειράματα καταλήξαμε στο συμπέρασμα ότι οι αποκρίσεις δεν έχουν αλλαγές μεταξύ τους, ο πιο ικανοποιητικός έλεγχος θα μπορούσαμε να πούμε ότι είναι αυτός που δεν περιλαμβάνει το διαφορικό όρο, καθώς δεν προσφέρει κάτι παραπάνω. 

\begin{figure}[h]
  \centering
  \includegraphics[width=\textwidth,height=5cm,keepaspectratio]{cruise_control_disturbances}
  \caption{Αντιμετώπιση διαταραχών του αυτο--ρυθμιζόμενου PID ελεγκτή του οποίου τα κέρδη έχουν υπολογιστεί με τους τύπους Ziegler -- Nichols}
  \label{fig:cruise_control_disturbances}
\end{figure}

\noindent
Στο Σχήμα \ref{fig:cruise_control_disturbances} φαίνεται πώς το κλειστό σύστημα αποκρίνεται στις ξαφνικές διαταραχές που εισέρχονται στο βρόχο ελέγχου. Εύκολα γίνεται αντιληπτό ότι οι διαταραχές αντιμετωπίζονται ικανοποιητικά αφού ο ελεγκτής δεν αφήνει την ελεγχόμενη μεταβλητή να αποκλίνει από την επιθυμητή τιμή της και το σύστημα επιστρέφει στην επιθυμητή τιμή όταν οι διαταραχές σταματάνε. 

%\begin{figure}[h]
%  \centering
%  \includegraphics[width=\textwidth,height=5cm,keepaspectratio]{cruise_control_disturbances_TL}
%  \caption{Αντιμετώπιση διαταραχών του αυτο--ρυθμιζόμενου PID ελεγκτή του οποίου τα κέρδη έχουν υπολογιστεί με τους τύπους Tyreus -- Luyben}
%  \label{fig:cruise_control_disturbances_TL}
%\end{figure}

\subsection{Συμπεράσματα}

Σε αυτή την ενότητα έγινε προσπάθεια να ελεγχθεί ένα σύστημα που συναντάται σε καθημερινή βάση σε όλα τα σύγχρονα οχήματα. Αρχικά, με μία απλή προσομοίωση κατέστη εμφανές ότι το σύστημα χρήζει ελέγχου, αφού χωρίς ελεγκτή η απόκριση ήταν πολύ αργή και η έξοδος του συστήματος είχε τιμή περίπου $y(t) = 0.1$ αντί για $y(t) = 5$ που ήταν το επιθυμητό. Αυτό είναι λογικό, αφού η συνάρτηση μεταφοράς (Εξίσωση \ref{eq:cruise_control_laplace}) για τις δοθείσες τιμές των παραμέτρων έχει κέρδος ανοιχτού βρόχου $\displaystyle \frac{1}{50} = 0.02$.

Προχωρώντας λοιπόν στον έλεγχο του συστήματος παρατηρήθηκε κάτι αρκετά ενδιαφέρον. Ασχέτως με τον τύπο ελέγχου που χρησιμοποιήθηκε ή τη μέθοδο με την οποία υπολογίστηκαν τα κέρδη, η απόκριση του κλειστού συστήματος φαίνεται να μένει ανεπηρέαστη. Αυτό οφείλεται τόσο στην ίδια τη φύση του εν λόγω συστήματος όσο και στον τρόπο που έχει μοντελοποιηθεί. Το συγκεκριμένο σύστημα είναι ένα αμάξι που πρέπει να διατηρεί σταθερή ταχύτητα. Όπως είναι φυσικό, η μάζα του οχήματος έχει μεγάλη τιμή, $m = 1000\ kg$ για το συγκεκριμένο παράδειγμα. Αυτό συνεπάγεται ότι ο έλεγχος θα πρέπει να έχει πολύ μεγάλη τιμή προκειμένου να προκαλέσει τις επιθυμητές αλλαγές στην ταχύτητα του αμαξιού.

Σε μια πραγματική εφαρμογή, στην οποία υπάρχουν περιορισμοί λόγω των φυσικών ορίων των συσκευών που χρησιμοποιούνται, η τιμή του ελέγχου θα πρέπει να περιορίζεται έτσι ώστε να αποφεύγεται η ταλαιπωρία και ενδεχομένως και η καταστροφή του εξοπλισμού. Στην εργασία αυτή, για την προσομοίωση ενός πραγματικού συστήματος ελέγχου, η έξοδος του ελεγκτή δεν μπορεί να ξεπερνάει τις τιμές $1000$ και $-1000$. Συνεπώς, στο συγκεκριμένο σύστημα ελέγχου, λόγω της μεγάλης μάζας του οχήματος, η έξοδος του ελεγκτή έρχεται σε κορεσμό, οποιοδήποτε είδος ελέγχου και να χρησιμοποιείται. Αυτό φαίνεται στα σχήματα \ref{fig:cruise_control_proportional}, \ref{fig:cruise_control_integral}, \ref{fig:cruise_control_derivative} και \ref{fig:cruise_control_TL} στη κάτω γραφική παράσταση που δείχνει την είσοδο του συστήματος (έξοδος του ελεγκτή) σε σχέση με την έξοδο του συστήματος. Έτσι λοιπόν εξηγείται γιατί το σύστημα έχει παρόμοια απόκριση ανεξαρτήτως του ελέγχου που χρησιμοποιείται. 

Ενδεικτικά, στο Σχήμα \ref{fig:cruise_control_no_limits} φαίνεται η θεωρητική απόκριση που θα είχε το σύστημα αν δεν υπήρχαν τα προαναφερθέντα όρια στην έξοδο του ελεγκτή και για αναλογικό κέρδος $K_c = 5000$. Εύκολα παρατηρούμε ότι η απόκριση είναι πολύ διαφορετική από αυτές που είδαμε προηγουμένως. Το σύστημα χρειάζεται μόνο μισό δευτερόλεπτο για να φτάσει στο $90\%$ της τελικής τιμής του σε αντίθεση με τα περίπου έξι δευτερόλεπτα που χρειαζόταν όταν το σήμα ελέγχου περιοριζόταν. Αυτό βέβαια για να επιτευχθεί χρειάζεται μια τεράστια τιμή του σήματος ελέγχου, συγκεκριμένα η τιμή αυτή για τη δοθείσα τιμή του αναλογικού κέρδους είναι $controller\ output = 50000$. Όπως εξηγήθηκε, μια τέτοια τιμή σε ένα πραγματικό σύστημα θα μπορούσε να έχει ολέθριες συνέπειες για τον εξοπλισμό, συνεπώς μια τέτοια απόκριση παραμένει πραγματοποιήσιμη μόνο σε θεωρητικό επίπεδο.

\begin{figure}[h]
  \centering
  \includegraphics[width=\textwidth,height=5cm,keepaspectratio]{cruise_control_no_limits}
  \caption{Βηματική απόκριση του συστήματος Αυτόματου Ελέγχου Ταχύτητας χωρίς όριο για την έξοδο του ελεγκτή}
  \label{fig:cruise_control_no_limits}
\end{figure}

Επειδή το σύστημα είναι πρώτης τάξης, οι ταλαντώσεις που εκτελεί κατά τη διαδικασία ρύθμισης έχουν πολύ μικρό πλάτος και περίοδο με αποτέλεσμα τα κέρδη που υπολογίζονται να είναι πολύ υψηλά. Αυτό όμως εν τέλει δεν επηρεάζει την απόκριση του συστήματος καθώς η έξοδος του ελεγκτή λειτουργεί μέσα σε συγκεκριμένες τιμές. Συνοψίζοντας, δεδομένου των περιορισμών που αναφέρθηκαν, ο αυτο--ρυθμιζόμενος PID ελεγκτής προσφέρει μια αρκετά ικανοποιητική μορφή ελέγχου.

\section{Σύστημα Motor Speed} \label{sec:motor_speed}

\subsection{Εισαγωγή}

Ένας κοινός ενεργοποιητής στα συστήματα ελέγχου είναι ο DC κινητήρας (\emph{κινητήρας συνεχούς ρεύματος}). Αυτός παρέχει άμεσα περιστροφική κίνηση και σε συνδυασμό με με τροχούς και άλλα μέσα μπορεί να προσφέρει και μεταφορική κίνηση. Το ηλεκτρικό ισοδύναμο ενός  DC κινητήρα και το διάγραμμα ελευθέρου σώματος φαίνονται στο παρακάτω σχήμα. 

\begin{figure}[h]
  \centering
  \includegraphics[width=\textwidth,height=5cm,keepaspectratio]{motor}
  \caption{Μοντέλο συστήματος ενός DC κινητήρα}
  \label{fig:motor}
\end{figure}

\subsection{Μαθηματικό Μοντέλο}

Για αυτό το παράδειγμα, θα θεωρήσουμε ότι η είσοδος του συστήματος είναι η τάση $V$ που εφαρμόζεται στο κύκλωμα και η έξοδος είναι η γωνιακή ταχύτητα $\dot{\theta}$ του άξονα. Ο ρότορας και ο άξονας υποτίθεται ότι είναι άκαμπτοι. Επίσης υποθέτουμε ότι η ροπή τριβής είναι ανάλογη της γωνιακής ταχύτητας του άξονα.

Γενικά, η ροπή που παράγεται από έναν κινητήρα συνεχούς ρεύματος είναι ανάλογη προς το ρεύμα οπλισμού και τη δύναμη του μαγνητικού πεδίου. Σε αυτό το παράδειγμα, θα υποθέσουμε ότι το μαγνητικό πεδίο είναι σταθερό και, συνεπώς, ότι η ροπή του κινητήρα είναι ανάλογη μόνο προς το ρεύμα $i$ του οπλισμού κατά ένα σταθερό παράγοντα $K_t$ όπως φαίνεται στην παρακάτω εξίσωση. Αυτό αναφέρεται ως κινητήρας ελεγχόμενος από οπλισμό.
\begin{equation}
T = K_ti
\end{equation}
Η πίσω ηλεκτροκινητική δύναμη (\emph{back emf}), $e$, είναι ανάλογη της γωνιακής ταχύτητας του άξονα με σταθερό παράγοντα $K_e$.
\begin{equation}
e = K_e\dot{\theta}
\end{equation}
Σε μονάδες SI, οι σταθερές του κινητήρα και της πίσω ηλεκτροκινητικής δύναμης είναι ίσες, δηλαδή, $K_t = K_e$. Ως εκ τούτου, θα χρησιμοποιήσουμε τη σταθερά $K$ για να αντιπροσωπεύσουμε τόσο τη σταθερά της ροπής του κινητήρα όσο και την σταθερά της πίσω ηλεκτροκινητικής δύναμης. Από το Σχήμα \ref{fig:motor}, μπορούμε να βρούμε τις ακόλουθες εξισώσεις που βασίζονται στον $2^o$ νόμο του Νεύτωνα και στον νόμο περί τάσης του Kirchhoff.
\begin{equation}
J\ddot{\theta} + b\dot{\theta} = Ki
\end{equation}
\begin{equation}
L\frac{di}{dt} + Ri = V - K\dot{\theta}
\end{equation}
Εφαρμόζοντας το μετασχηματισμό Laplace, οι παραπάνω εξισώσεις μοντελοποίησης μπορούν να εκφραστούν με τη μεταβλητή Laplace $s$.
\begin{equation}
s(Js+b)\Theta(s) = KI(s)
\end{equation}
\begin{equation}
(Ls+R)I(s) = V(s) - Ks\Theta(s)
\end{equation}
Εξαλείφοντας τον όρο $I(s)$ από τις δύο παραπάνω εξισώσεις καταλήγουμε στην ακόλουθη συνάρτηση μεταφοράς ανοιχτού συστήματος, όπου η γωνιακή ταχύτητα θεωρείται η έξοδος και η τάση θεωρείται η είσοδος
\begin{equation}
P(s) = \frac{\dot{\Theta}}{V(s)} = \frac{K}{(JL)s^2+(RJ+bL)s+(bR+K^2)} \left[\frac{rad/sec}{V}\right]
\label{eq:motor_laplace}
\end{equation}

\subsection{Πείραμα}

Για το παράδειγμα αυτό θεωρούμε ότι οι τιμές των παραμέτρων είναι ως εξής
\begin{flushleft}
\begin{tabular}{lll}
\textbf{J} & ροπή αδράνειας του ρότορα & $0.01\ kgm^2$ \\ 
\textbf{b} & σταθερά τριβής & $0.1\ Nms$ \\  
$\mathbf{K_e}$ & σταθερά ηλεκτροκινητικής δύναμης & $0.01\ \frac{V}{rad/sec}$ \\  
$\mathbf{K_t}$ & σταθερά ροπής του κινητήρα & $0.01\ \frac{Nm}{Ampere}$ \\  
\textbf{R} & ηλεκτρική αντίσταση & $1\ Ohm$ \\ 
\textbf{L} & ηλεκτρική επαγωγή & $0.5\ H$ \\ 
\end{tabular} 
\end{flushleft}

Αντικαθιστώντας τις τιμές αυτές στη σχέση \ref{eq:motor_laplace} έχουμε την παρακάτω εξίσωση
\begin{equation}
P_{motor} = \frac{0.01}{0.005s^2+0.06s+0.1001}
\end{equation}

\subsubsection{Απόκριση Χωρίς Έλεγχο}

Όπως και για τα προηγούμενα συστήματα, πρώτα δοκιμάζουμε την απόκριση του συστήματος όταν αυτό δουλεύει χωρίς έλεγχο.
\begin{figure}[h]
  \centering
  \includegraphics[width=\textwidth,height=5cm,keepaspectratio]{motor_no_control}
  \caption{Βηματική απόκριση του συστήματος ενός DC κινητήρα}
  \label{fig:motor_no_control}
\end{figure}
Από το παραπάνω σχήμα γίνεται εύκολα αντιληπτό ότι όταν το σύστημα λειτουργεί χωρίς έλεγχο η απόκριση του είναι κακή. Πιο συγκεκριμένα, βλέπουμε ότι για τάση εισόδου $1$ Volt, ο ρότορας στρέφεται με γωνιακή ταχύτητα $0.1\ \text{rad/sec}$ δηλαδή έχει δέκα φορές μικρότερη ταχύτητα από την επιθυμητή. Επίσης έχει χρόνο ανύψωσης περίπου δύο δευτερόλεπτα.

Επειδή το κυρίως ζητούμενο από ένα τέτοιο σύστημα είναι να περιστρέφεται στην επιθυμητή ταχύτητα, ο έλεγχος που θα εφαρμοστεί θέλουμε να έχει μικρό σφάλμα, μικρότερο από $1\%$. Άλλη απαίτηση ελέγχου είναι να έχει μικρό χρόνο ανύψωσης, έτσι ώστε να φτάνει σε σύντομο χρονικό διάστημα το επιθυμητό σημείο. Τέλος, λόγω του ότι η λειτουργία του κινητήρα σε ταχύτητα μεγαλύτερη της επιθυμητής μπορεί να δημιουργήσει προβλήματα στον εξοπλισμό, θέλουμε να έχει μικρό ποσοστό υπερακόντισης.

\subsubsection{Αναλογικός Έλεγχος}

Στο Σχήμα \ref{fig:motor_proportional} φαίνεται πώς συμπεριφέρεται το σύστημα όταν σε αυτό εφαρμόζεται μόνο αναλογικός έλεγχος, το κέρδος του οποίου έχει υπολογιστεί με τους τύπους Ziegler -- Nichols και η ταχύτητα απόκρισης του συστήματος έχει οριστεί στο ``Normal". Αμέσως γίνεται εμφανές ότι το σύστημα ενώ έχει ικανοποιητικό χρόνο ανύψωσης, παρουσιάζει μεγάλο ποσοστό υπερακόντισης και εκτεταμένες ταλαντώσεις μέχρι να βρεθεί σε κατάσταση ισορροπίας.

\begin{figure}[h]
  \centering
  \includegraphics[width=\textwidth,height=5cm,keepaspectratio]{motor_proportional}
  \caption{Βηματική απόκριση του συστήματος ενός DC κινητήρα με εφαρμογή αναλογικού ελέγχου}
  \label{fig:motor_proportional}
\end{figure}

Στο Σχήμα \ref{fig:motor_proportional_slow} φαίνεται ότι όταν η επιθυμητή ταχύτητα απόκρισης οριστεί στο ``Slow" τόσο οι ταλαντώσεις όσο και η υπερακόντιση μειώνονται. Συνεπώς για τη συνέχεια του πειράματος αυτή θα είναι η λειτουργία που θα χρησιμοποιηθεί.

\begin{figure}[h]
  \centering
  \includegraphics[width=\textwidth,height=5cm,keepaspectratio]{motor_proportional_slow}
  \caption{Βηματική απόκριση του συστήματος ενός DC κινητήρα με εφαρμογή αναλογικού ελέγχου και επιθυμητή ταχύτητα απόκρισης ``Slow"}
  \label{fig:motor_proportional_slow}
\end{figure}

\subsubsection{Αναλογικός  --  Διαφορικός Έλεγχος}

Συνεχίζοντας την προσπάθεια ελέγχου του συστήματος, προσθέτουμε και τον διαφορικό όρο. Το μεγαλύτερο πρόβλημα της προηγούμενης απόκρισης ήταν η μεταβατική κατάσταση του κινητήρα η οποία παρουσίαζε υψηλή υπερακόντιση και ταλαντώσεις. Για την αντιμετώπιση αυτών των φαινόμενων εισάγεται και ο διαφορικός έλεγχος. Η απόκριση του συστήματος παρουσιάζεται στο Σχήμα \ref{fig:motor_derivative_slow}.

Είναι εμφανές ότι ο διαφορικός όρος εξομάλυνε τη συμπεριφορά του συστήματος αλλά, όπως αναμενόταν, δεν μείωσε το σφάλμα μόνιμης κατάστασης το οποίο παραμένει στα ίδια επίπεδα με πριν.

\begin{figure}[h]
  \centering
  \includegraphics[width=\textwidth,height=5cm,keepaspectratio]{motor_derivative_slow}
  \caption{Βηματική απόκριση του συστήματος ενός DC κινητήρα με εφαρμογή αναλογικού  --  διαφορικού ελέγχου και επιθυμητή ταχύτητα απόκρισης ``Slow"}
  \label{fig:motor_derivative_slow}
\end{figure}

\subsubsection{Αναλογικός  --  Ολοκληρωτικός  --  Διαφορικός Έλεγχος}

Από τη γραφική παράσταση που φαίνεται στο Σχήμα \ref{fig:motor_pid_slow} βλέπουμε ότι πλέον έχουν εξομαλυνθεί τα αρνητικά χαρακτηριστικά της μεταβατικής κατάστασης και το μόνιμο σφάλμα είναι μηδενικό.

\begin{figure}[h]
  \centering
  \includegraphics[width=\textwidth,height=5cm,keepaspectratio]{motor_pid_slow}
  \caption{Βηματική απόκριση του συστήματος ενός DC κινητήρα με εφαρμογή αναλογικού  --  ολοκληρωτικού  --  διαφορικού ελέγχου και επιθυμητή ταχύτητα απόκρισης ``Slow"}
  \label{fig:motor_pid_slow}
\end{figure}

Επίσης, η απόκριση του κινητήρα όταν αυτός ξεκινάει από την ηρεμία φαίνεται στο Σχήμα \ref{fig:motor_pid_slow_start}. Βλέπουμε ότι αυτή η απόκριση διαφέρει λίγο από την απόκριση που έχει ο κινητήρας όταν τη θέση του relay στοιχείου παίρνει ο ελεγκτής. Αυτό οφείλεται στις διαφορετικές αρχικές συνθήκες που υπάρχουν στο σύστημα. Όταν αυτό ξεκινάει από την ηρεμία οι αρχικές του συνθήκες είναι μηδενικές, ενώ όταν υπόκειται στο πείραμα για τον υπολογισμό των κερδών του αυτό φυσικά δεν ισχύει.

\begin{figure}[h]
  \centering
  \includegraphics[width=\textwidth,height=5cm,keepaspectratio]{motor_pid_slow_start}
  \caption{Βηματική απόκριση του συστήματος DC κινητήρα όταν αυτός ξεκινάει από την ηρεμία και ελέγχεται με τη χρήση του αυτο--ρυθμιζόμενου PID ελεγκτή}
  \label{fig:motor_pid_slow_start}
\end{figure}

\subsubsection{Αντιμετώπιση Διαταραχών}

Στο Σχήμα \ref{fig:motor_disturbances_slow} φαίνεται πώς το κλειστό σύστημα ελέγχου διαχειρίζεται τις διαταραχές στην είσοδό της συνάρτησης μεταφοράς. Ο ελεγκτής κάνει καλή δουλειά στην αντιμετώπιση του θορύβου που μπορεί να υπεισέρχεται στο σύστημα του DC κινητήρα, κρατώντας την ταχύτητά του κοντά στα επιθυμητά πλαίσια και επαναφέροντάς την γρήγορα όταν οι διαταραχές σταματήσουν.

\begin{figure}[h]
  \centering
  \includegraphics[width=\textwidth,height=5cm,keepaspectratio]{motor_disturbances_slow}
  \caption{Αντιμετώπιση διαταραχών του αυτο--ρυθμιζόμενου PID ελεγκτή του οποίου τα κέρδη έχουν υπολογιστεί με τους τύπους Ziegler -- Nichols}
  \label{fig:motor_disturbances_slow}
\end{figure}

\subsection{Συμπεράσματα}

Σε αυτή την ενότητα χρησιμοποιώντας τον αυτο-ρυθμιζόμενο PID ελεγκτή επιχειρήσαμε να ελέγξουμε την ταχύτητα ενός ηλεκτρικού κινητήρα συνεχούς ρεύματος.

Αρχικά διαπιστώσαμε ότι χωρίς έλεγχο ο κινητήρας έχει πολύ χαμηλή απόδοση, με τελική γωνιακή ταχύτητα ίση με το ένα δέκατο της επιθυμητής. Στη συνέχεια, προστέθηκε ο αναλογικός όρος του ελεγκτή προκειμένου να δούμε αν αυτός αρκεί για να ελέγξει το σύστημα. Παρόλο που με αυτή την προσθήκη η τελική τιμή της ταχύτητας του ρότορα ήταν πολύ πιο κοντά στην επιθυμητή, παρουσιάζοντας σφάλμα $4.432\%$, οι ταλαντώσεις και η υπερακόντιση που παρουσίασε το σύστημα κατά τη μεταβατική του κατάσταση είναι μη αποδεκτά φαινόμενα, οπότε η χρήση μόνο αναλογικού ελεγκτή απορρίφθηκε σαν λύση ικανοποιητικού ελέγχου.

Φυσικό επακόλουθο ήταν να χρησιμοποιηθεί και ο διαφορικός όρος του ελεγκτή ο οποίος βελτιώνει την ευστάθεια του συστήματος και εξομαλύνει τα αρνητικά φαινόμενα που αναφέρθηκαν. Πράγματι, το διαφορικό κέρδος που υπολόγισε ο αυτο--ρυθμιζόμενος ελεγκτής ήταν ικανό να οδηγήσει σε μια ομαλή πορεία της εξόδου του συστήματος προς το επιθυμητό σημείο. Όμως, δεν είχε καμιά επιρροή στο σφάλμα μόνιμης κατάστασης το οποίο παρέμεινε στην ίδια τιμή.

Σε μια εφαρμογή σαν αυτή, το να πιάσει το σύστημα την είσοδο αναφοράς είναι ίσως το πιο σημαντικό κριτήριο ελέγχου, συνεπώς το σφάλμα μόνιμης κατάστασης έπρεπε να εξαλειφθεί. Οπότε, εισάγαμε στον έλεγχο και τον ολοκληρωτικό όρο του ελεγκτή έτσι ώστε να μηδενιστεί αυτό το σφάλμα. Με τη χρήση και των τριών όρων του, ο αυτο--ρυθμιζόμενος PID ελεγκτής κατάφερε επιτυχώς να καλύψει όλες τις απαιτήσεις ελέγχου. Το σύστημα δεν παρουσιάζει ταλαντώσεις ή υπερακόντιση, έχει μικρό χρόνο ανύψωσης, αντιμετωπίζει ικανοποιητικά τις διαταραχές και έχει μηδενικό σφάλμα μόνιμης κατάστασης.

\section{Σύστημα Inverted Pendulum} \label{sec:inverted_pendulum}

\subsection{Εισαγωγή}

Το σύστημα σε αυτό το παράδειγμα αποτελείται από ένα ανεστραμμένο εκκρεμές τοποθετημένο σε ένα μηχανοκίνητο καροτσάκι. Το σύστημα ανεστραμμένου εκκρεμούς συνήθως συναντάται σε πολλά βιβλία των συστημάτων αυτομάτου ελέγχου καθώς και στην ερευνητική βιβλιογραφία. Η δημοτικότητά του απορρέει εν μέρει από το γεγονός ότι είναι ασταθές χωρίς έλεγχο, δηλαδή το εκκρεμές απλά θα πέσει αν το καλάθι δεν μετακινηθεί για να το ισορροπήσει. Επιπλέον, η δυναμική του συστήματος είναι μη γραμμική. Ο στόχος του συστήματος ελέγχου είναι να ισορροπήσει το ανεστραμμένο εκκρεμές εφαρμόζοντας μια δύναμη στο καλάθι με το εκκρεμές. Ένα παράδειγμα πραγματικού κόσμου που σχετίζεται άμεσα με αυτό το σύστημα ανεστραμμένου εκκρεμούς είναι ο έλεγχος θέσης ενός πυραύλου εκτόξευσης κατά την απογείωση.

Σε αυτή την περίπτωση, θα εξετάσουμε ένα δισδιάστατο πρόβλημα όπου το εκκρεμές είναι περιορισμένο να μετακινείται στο κατακόρυφο επίπεδο που φαίνεται στο παρακάτω σχήμα. Υπό κανονικές συνθήκες, το σύστημα του ανεστραμμένου εκκρεμούς αποτελεί ένα σύστημα μίας εισόδου, δύο εξόδων. Όμως ο PID ελεγκτής μπορεί να ελέγξει συστήματα μίας εισόδου, μίας εξόδου (\emph{Single Input  --  Single Output} ή \emph{SISO}). Συνεπώς, για το σύστημα αυτό, η είσοδος ελέγχου είναι η δύναμη που κινεί το καλάθι με οριζόντιο τρόπο και η έξοδός του είναι η γωνιακή θέση του εκκρεμούς.

\begin{figure}[h]
  \centering
  \includegraphics[width=\textwidth,height=5cm,keepaspectratio]{pendulum}
  \caption{Μοντέλο του συστήματος Ανεστραμμένου Εκκρεμούς}
  \label{fig:pendulum}
\end{figure}

\subsection{Μαθηματικό Μοντέλο}

Στο παρακάτω σχήμα φαίνεται το διάγραμμα ελευθέρου σώματος για το σύστημα ανεστραμμένου εκκρεμούς.

\begin{figure}[h]
  \centering
  \includegraphics[width=\textwidth,height=5cm,keepaspectratio]{pendulum2}
  \caption{Διάγραμμα ελευθέρου σώματος για το σύστημα ανεστραμμένου εκκρεμούς}
  \label{fig:pendulum2}
\end{figure}

Συγκεντρώνοντας τις δυνάμεις στο διάγραμμα ελεύθερου σώματος του καροτσιού στην οριζόντια κατεύθυνση, έχουμε την ακόλουθη εξίσωση κίνησης
\begin{equation}
M\ddot{x} + b\dot{x} + N = F
\end{equation}
Συγκεντρώνοντας τις δυνάμεις στο διάγραμμα ελεύθερου σώματος του εκκρεμούς στην οριζόντια κατεύθυνση, λαμβάνουμε την ακόλουθη έκφραση για τη δύναμη αντίδρασης $N$
\begin{equation}
N = m\ddot{x} + ml\ddot{\theta}\cos{\theta} - ml\dot{\theta}^2\sin{\theta}
\end{equation}
Αν αντικαταστήσουμε αυτή την εξίσωση στην πρώτη εξίσωση, παίρνουμε μια από τις δύο εξισώσεις που ισχύουν για αυτό το σύστημα
\begin{equation}
(M+m)\ddot{x} + b\dot{x} + ml\ddot{\theta}\cos{\theta}-ml\dot{\theta}^2\sin{\theta} = F
\end{equation}
Για να πάρουμε τη δεύτερη εξίσωση κίνησης για αυτό το σύστημα, αθροίζουμε τις δυνάμεις κάθετες στο εκκρεμές. Η εξίσωση που προκύπτει είναι η ακόλουθη
\begin{equation}
P\sin{\theta}+N\cos{\theta}-mg\sin{\theta}=ml\ddot{\theta}+m\ddot{x}\cos{\theta}
\end{equation}
Για να απαλλαγούμε από τους όρους της παραπάνω εξίσωσης, αθροίζουμε τις ροπές του εκκρεμούς για να πάρουμε την ακόλουθη εξίσωση
\begin{equation}
-Pl\sin{\theta}-Nl\cos{\theta}=I\ddot{\theta}
\end{equation}
Συνδυάζοντας αυτές τις δύο τελευταίες εκφράσεις, παίρνουμε τη δεύτερη εξίσωση κίνησης
\begin{equation}
\left(I+ml^2\right)\ddot{\theta}+mgl\sin{\theta}=-ml\ddot{x}\cos{\theta}
\end{equation}
Δεδομένου ότι οι τεχνικές ανάλυσης και σχεδιασμού ελέγχου που θα χρησιμοποιήσουμε σε αυτό το παράδειγμα ισχύουν μόνο για γραμμικά συστήματα, αυτό το σύνολο εξισώσεων πρέπει να γραμμικοποιηθεί. Συγκεκριμένα, θα γραμμικοποιήσουμε τις εξισώσεις σχετικά με την κατακόρυφη πάνω θέση $\theta=\pi$, και θα υποθέσουμε ότι το σύστημα παραμένει μέσα σε μια μικρή γειτονιά αυτής της ισορροπίας. Αυτή η παραδοχή θα πρέπει να είναι λογικά έγκυρη αφού υπό τον έλεγχο θέλουμε το εκκρεμές να μην αποκλίνει πολύ από την κάθετα προς τα άνω θέση. Έστω ότι $\phi$ είναι η απόκλιση της θέσης του εκκρεμούς από την ισορροπία, δηλαδή, $\theta=\pi+\phi$. Συνεπώς, υποθέτωντας μια μικρή απόκλιση, $\phi$, από την ισορροπία, μπορούμε να χρησιμοποιήσουμε τις παρακάτω προσεγγίσεις μικρών γωνιών των μη γραμμικών συναρτήσεων στις εξισώσεις του συστήματός μας
\begin{equation}
\cos\theta=\cos(\pi+\phi)\approx -1
\end{equation}
\begin{equation}
\sin\theta=\sin(\pi+\phi)\approx -\phi
\end{equation}
\begin{equation}
\dot{\theta}^2=\dot{\phi}^2\approx 0
\end{equation}
Αφού αντικαταστήσουμε τις παραπάνω προσεγγίσεις στις μη γραμμικές εξισώσεις μας, φτάνουμε στις δύο γραμμικές εξισώσεις κίνησης. Το σύμβολο της εισόδου $F$ αντικαταστάθηκε από το σύμβολο $u$.
\begin{equation}
\left(I+ml^2\right)\ddot{\phi}-mgl\phi=ml\ddot{x}
\end{equation}
\begin{equation}
\left(M+m\right)\ddot{x}+b\dot{x}-ml\ddot{\phi}=u
\end{equation}
Για να λάβουμε τις συναρτήσεις μεταφοράς των εξισώσεων του γραμμικού συστήματος, πρέπει πρώτα να πάρουμε το μετασχηματισμό Laplace των εξισώσεων αυτών, υποθέτοντας μηδενικές αρχικές συνθήκες. Οι μετασχηματισμοί Laplace που προκύπτουν παρουσιάζονται παρακάτω.
\begin{equation}
\left(I+ml^2\right)\Phi(s)s^2-mgl\Phi(s)=mlX(s)s^2
\end{equation}
\begin{equation}
\left(M+m\right)X(s)s^2+bX(s)s-ml\Phi(s)s^2=U(s)
\label{eq:inverted_pendulum_5_28}
\end{equation}
Όπως είναι γνωστό, μια συνάρτηση μεταφοράς αντιπροσωπεύει τη σχέση μεταξύ μιας μόνο εισόδου και μιας μόνο εξόδου κάθε φορά. Για να βρούμε τη συνάρτηση μεταφοράς για την έξοδο $\Phi(s)$ και της εισόδου $U(s)$ πρέπει να εξαλείψουμε το $X(s)$ από τις παραπάνω εξισώσεις. Λύνοντας την πρώτη εξίσωση ως προς το $X(s)$ έχουμε
\begin{equation}
X(s) = \left[\frac{I+ml^2}{ml}-\frac{g}{s^2}\right]\Phi(s)
\end{equation}
Αντικαθιστώντας αυτή την εξίσωση στην εξίσωση \ref{eq:inverted_pendulum_5_28} έχουμε
\begin{equation}
\left(M+m\right)\left[\frac{I+ml^2}{ml}\frac{g}{s^2}\right]\Phi(s)s^2+b\left[\frac{I+ml^2}{ml}\frac{g}{s^2}\right]\Phi(s)s-ml\Phi(s)s^2=U(s)
\end{equation}
Αναδιατάσσοντας την εξίσωση αυτή παίρνουμε τη συνάρτηση μεταφοράς
\begin{equation}
\frac{\Phi(s)}{U(s)} = \frac{\frac{ml}{q}s}{s^3+\frac{b(I+ml^2)}{q}s^2-\frac{(M+m)mgl}{q}s-\frac{bmgl}{q}} \left[\frac{rad}{N}\right]
\end{equation}
όπου,
\begin{equation}
q = \left(M+m\right)\left(I+ml^2\right)-\left(ml\right)^2
\end{equation}

\subsection{Πείραμα}

Υποθέτουμε ότι για αυτό το παράδειγμα οι παράμετροι του συστήματος είναι ως εξής
\begin{flushleft}
\begin{tabular}{lll}
$\mathbf{M}$ & μάζα του καροτσιού & $0.5\ kg$ \\  
$\mathbf{m}$ & μάζα του εκκρεμούς & $0.2\ kg$ \\ 
$\mathbf{b}$ & σταθερά τριβής για το καρότσι & $0.1 \frac{N}{m/s}$ \\ 
$\mathbf{l}$ & απόσταση του κέντρου μάζας του εκκρεμούς & $0.3\ m$ \\ 
$\mathbf{I}$ & ροπή αδράνειας του εκκρεμούς & $0.006\ kgm^2$ \\ 
\end{tabular}
\end{flushleft}
 
Το πείραμα αυτή τη φορά είναι λίγο διαφορετικό από τα προηγούμενα. Όπως αναφέρθηκε, το συγκεκριμένο σύστημα κανονικά έχει μία είσοδο και δύο εξόδους. Για να προσαρμοστεί όμως στον PID έλεγχο αγνοήσαμε τη μία έξοδο που αποτελούσε τη θέση του καροτσιού. Ως αποτέλεσμα και η πειραματική διαδικασία θα πρέπει να υπόκειται σε κάποιους περιορισμούς.

Πιο συγκεκριμένα, θα θεωρήσουμε ότι το εκκρεμές είναι ήδη ακίνητο στην κατακόρυφη θέση $\theta = \pi$. Αυτό σημαίνει ότι η επιθυμητή τιμή είναι setpoint $=0$. Για τον έλεγχο του συστήματος θα σπρώχνουμε στιγμιαία το καρότσι, δηλαδή θα εισάγουμε μια κρουστική με τη μορφή διαταραχής στο σύστημα. Στόχος του ελέγχου θα είναι ο αυτο--ρυθμιζόμενος PID ελεγκτής να υπολογίσει τα κέρδη έτσι ώστε το εκκρεμές να επανέρχεται στην κατακόρυφη θέση χωρίς να απομακρύνεται πολύ από αυτή. 

\begin{figure}[h]
  \centering
  \includegraphics[width=\textwidth,height=5cm,keepaspectratio]{pendulum_relay}
  \caption{Απόκριση του συστήματος κατά τη διάρκεια του relay πειράματος}
  \label{fig:pendulum_relay}
\end{figure}

Επίσης, προκειμένου να υπολογίσει ο αλγόριθμος το κατάλληλο αναλογικό κέρδος το σύστημα θα πρέπει να υποστεί τη διαδικασία των ταλαντώσεων λόγω του relay στοιχείου. Επειδή το σύστημα είναι από μόνο του ασταθές, οι ταλαντώσεις αυτές δε θα έχουν σταθερό πλάτος οδηγώντας μετά από λίγο το σύστημα στην αστάθεια (βλέπε Σχήμα \ref{fig:pendulum_relay}). Προκειμένου λοιπόν να βρει ο αλγόριθμος κάποιες τιμές για τα κέρδη το πείραμα θα διακόπτεται πριν η απόκριση του συστήματος τείνει στο άπειρο. 



\subsubsection{Απόκριση Χωρίς Έλεγχο}

\begin{figure}[h]
  \centering
  \includegraphics[width=\textwidth,height=5cm,keepaspectratio]{pendulum_no_control}
  \caption{Απόκριση του συστήματος Ανεστραμμένου Εκκρεμούς χωρίς έλεγχο}
  \label{fig:pendulum_no_control}
\end{figure}

Από το παραπάνω σχήμα γίνεται εύκολα αντιληπτό ότι το σύστημα όταν λειτουργεί χωρίς έλεγχο είναι ασταθές. Μια μικρή διαταραχή έχει ως αποτέλεσμα η γωνία απόκλισης του εκκρεμούς να τείνει στο άπειρο.

\subsubsection{Αναλογικός Έλεγχος}

Τώρα θα γίνει προσπάθεια να ελεγχθεί το σύστημα χρησιμοποιώντας μόνο τον αναλογικό όρο. Η προσπάθεια αυτή αναπαρίσταται στο Σχήμα \ref{fig:pendulum_proportional}.

\begin{figure}[h]
  \centering
  \includegraphics[width=\textwidth,height=5cm,keepaspectratio]{pendulum_proportional}
  \caption{Απόκριση του συστήματος ανεστραμμένου εκκρεμούς με εφαρμογή αναλογικού ελέγχου}
  \label{fig:pendulum_proportional}
\end{figure}
Μετά το πέμπτο δευτερόλεπτο διακόπτουμε το πείραμα relay. Το αναλογικό κέρδος που έχει υπολογιστεί είναι $K_c=13.475$. Αυτό το κέρδος δεν είναι ικανό να επαναφέρει το εκκρεμές στην αρχική του θέση, αλλά ούτε το ωθεί στην αστάθεια αφού εκτελεί αμείωτες ταλαντώσεις. 

\subsubsection{Αναλογικός  --  Διαφορικός Έλεγχος}

Στη συνέχεια προστίθεται και ο διαφορικός όρος προκειμένου να βελτιωθεί η ευστάθεια του συστήματος και να μετριαστούν οι ταλαντώσεις. Επίσης, οι τύποι για τον υπολογισμό των κερδών για τη συνέχεια του πειράματος θα είναι αυτοί που βασίζονται στη μέθοδο Tyreus -- Luyben επειδή προσφέρουν μια λιγότερο ``επιθετική" προσέγγιση στον έλεγχο του συστήματος. Η απόκριση του συστήματος φαίνεται στο παρακάτω σχήμα.

\begin{figure}[h]
  \centering
  \includegraphics[width=\textwidth,height=5cm,keepaspectratio]{pendulum_pd}
  \caption{Απόκριση του συστήματος ανεστραμμένου εκκρεμούς με εφαρμογή αναλογικού  --  διαφορικού ελέγχου}
  \label{fig:pendulum_pd}
\end{figure}

Η βελτίωση στην απόκριση του συστήματος είναι εμφανής. Οι τιμές των όρων του ελεγκτή που υπολογίστηκαν από το πείραμα relay ($K_c = 13.553,\ T_d = 0,000952$) είναι ικανές να φέρουν το σύστημα πολύ κοντά στην αρχική του θέση. Επίσης οι κρουστικές διαταραχές δεν το απομακρύνουν πολύ από τη θέση ισορροπίας του. Όμως με μια προσεκτική ματιά παρατηρούμε ότι η γωνία του εκκρεμούς ολοένα και αυξάνεται, αργά αλλά σταθερά. Συνεπώς ούτε τώρα το σύστημα είναι απολύτως ευσταθές, απλά αργεί πολύ περισσότερο να οδηγηθεί στην αστάθεια.

\subsubsection{Αναλογικός  --  Ολοκληρωτικός  --  Διαφορικός Έλεγχος}

\begin{figure}[h]
  \centering
  \includegraphics[width=\textwidth,height=5cm,keepaspectratio]{pendulum_pid}
  \caption{Απόκριση του συστήματος ανεστραμμένου εκκρεμούς με εφαρμογή αναλογικού  --  ολοκληρωτικού  --  διαφορικού ελέγχου}
  \label{fig:pendulum_pid}
\end{figure}

Στο παρακάτω σχήμα φαίνεται η απόκριση του συστήματος όταν εφαρμόζεται PID έλεγχος. Πλέον ο ελεγκτής είναι ικανός να επαναφέρει το σύστημα στην αρχική του θέση. Επίσης, όταν εφαρμόζονται κρουστικές διαταραχές ο έλεγχος δεν αφήνει το εκκρεμές να απομακρυνθεί πολύ από τη θέση ισορροπίας και η επαναφορά του σε αυτή γίνεται σε μικρό χρονικό διάστημα. 

\subsection{Συμπεράσματα}

Σε αυτή την ενότητα ο αυτο--ρυθμιζόμενος PID ελεγκτής επιχείρησε να ελέγξει ένα αρκετά απαιτητικό σύστημα. Το ανεστραμμένο εκκρεμές είναι ένα εν γένει μη γραμμικό, ασταθές σύστημα, μίας εισόδου -- δύο εξόδων. Αυτό οδήγησε σε αρκετούς συμβιβασμούς στον έλεγχο.

Αρχικά το σύστημα έπρεπε να γραμμικοποιηθεί γύρω από την κατακόρυφη θέση ισορροπίας του $\theta = \pi$. Ως αποτέλεσμα αυτού, η συνάρτηση μεταφοράς του ισχύει μόνο για μικρές αποκλίσεις της γωνίας του εκκρεμούς από την αρχική του θέση. Επίσης, επειδή ο PID έλεγχος χρησιμοποιείται για έλεγχο συστημάτων μίας εισόδου -- μίας εξόδου, αγνοήσαμε εντελώς τη θέση του καροτσιού και λάβαμε υπόψιν μας μόνο τη γωνία του εκκρεμούς. Τέλος, λόγω της εκ φύσεως αστάθειας του συστήματος, το πείραμα relay μέσω του οποίου ο αλγόριθμος εκτιμά τις τιμές των όρων του ελεγκτή δεν οδηγούσε σε σταθερές αμείωτες ταλαντώσεις οπότε έπρεπε να διακόπτεται πριν το σύστημα οδηγηθεί στην αστάθεια. 

Όμως, παρόλες τις δυσκολίες ελέγχου, ο αυτο--ρυθμιζόμενος PID ελεγκτής πρόσφερε μια ικανοποιητική λύση για το συγκεκριμένο πρόβλημα. Παρότι τα κέρδη υπολογίζονται χρησιμοποιώντας ταλαντώσεις μη σταθερού πλάτους, οι τιμές που τους αποδίδονται κρίνονται επαρκής για τις απαιτήσεις ελέγχου ενός τέτοιου συστήματος. Με τη χρήση και των τριών όρων του δεν αφήνει το εκκρεμές να αποκλίνει πολύ από τη θέση ισορροπίας του όταν υπεισέρχονται διαταραχές στο σύστημα. Επίσης το επαναφέρει στην αρχική του θέση σε λιγότερο από πέντε δευτερόλεπτα και εμφανίζει μηδενικό σφάλμα μόνιμης κατάστασης. Με λίγα λόγια το σύστημα πλέον είναι ευσταθές, με ανοχή σε στιγμιαίες διαταραχές και συνεπώς η αυτόματη ρύθμιση του PID ελεγκτή κρίνεται επιτυχής.

\section{Σύστημα Aircraft Pitch} \label{sec:aircraft_pitch}

\begin{figure}[h]
  \centering
  \includegraphics[width=\textwidth,height=5cm,keepaspectratio]{flightdynamics}
  \caption{Μοντέλο του συστήματος Αεροσκάφους}
  \label{fig:flightdynamics}
\end{figure}

\subsection{Εισαγωγή}

Σε αυτή την ενότητα θα γίνει προσπάθεια να σχεδιασθεί ένας αυτόματος πιλότος που θα ελέγχει το γωνιακό βήμα ενός αεροσκάφους (\emph{aircraft pitch}).

\subsection{Μαθηματικό Μοντέλο}
Οι βασικοί άξονες και οι δυνάμεις των συντεταγμένων που δρουν σε ένα αεροσκάφος παρουσιάζονται στο Σχήμα \ref{fig:flightdynamics}. Οι εξισώσεις που διέπουν την κίνηση ενός αεροσκάφους είναι ένα πολύ περίπλοκο σύνολο έξι μη γραμμικών συζευγμένων διαφορικών εξισώσεων. Ωστόσο, με ορισμένες υποθέσεις, μπορούν να αποσυνδεθούν και να γραμμικοποιηθούν σε διαμήκεις και πλευρικές εξισώσεις. Σύμφωνα με αυτές τις υποθέσεις, οι διαμήκεις εξισώσεις κίνησης του αεροσκάφους μπορούν να γραφτούν ως εξής
\begin{equation}
\dot{\alpha}=\mu\Omega\sigma\left[-\left(C_L+C_D\right)\alpha+\frac{1}{\mu-C_L}q-\left(C_W\sin\gamma\right)\theta+C_L\right]
\end{equation}
\begin{equation}
\dot{q}=\frac{\mu\Omega}{2i_{yy}}\left[\left[C_M-\eta\left(C_L+C_D\right)\right]\alpha+\left[C_M+\sigma C_M\left(1-\mu C_L\right)\right]q+\left(\eta C_W\sin\gamma\right)\delta\right]
\end{equation}
\begin{equation}
\dot{\theta}=\Omega q
\end{equation}
Το πώς προκύπτουν αυτές οι εξισώσεις μπορεί να βρεθεί σε οποιοδήποτε διδακτικό βιβλίο σχετικά με αεροσκάφη. Για το σύστημα αυτό, η είσοδος θα είναι η γωνία κλίσης $\delta$ και η έξοδος θα είναι η γωνία βήματος $\theta$ του αεροσκάφους.

Πριν από την εύρεση της συνάρτησης μεταφοράς, ας αντικαταστήσουμε μερικές αριθμητικές τιμές στις παραμέτρους του συστήματος για να απλοποιήσουμε τις παραπάνω εξισώσεις μοντελοποίησης:
\begin{equation}
\dot{\alpha} = -0.313\alpha + 56.7q + 0.232\delta
\end{equation}
\begin{equation}
\dot{q} = -0.0139\alpha - 0.426q + 0.0203\delta
\end{equation}
\begin{equation}
\dot{\theta} = 56.7q
\end{equation}
Αυτές οι τιμές λαμβάνονται από τα δεδομένα ενός εμπορικού αεροσκάφους της Boeing.

Για να βρούμε τη συνάρτηση μεταφοράς του συστήματος, πρέπει να πάρουμε το μετασχηματισμό Laplace των παραπάνω εξισώσεων μοντελοποίησης. Για την εύρεση της συνάρτησης μεταφοράς θεωρούμε μηδενικές αρχικές συνθήκες. Ο μετασχηματισμός Laplace των παραπάνω εξισώσεων παρουσιάζεται παρακάτω
\begin{equation}
sA(s)=-0.313A(s)+56.7Q(s)+0.232\Delta(s)
\end{equation}
\begin{equation}
sQ(s)=-0.0139A(s)-0.426Q(s)+0.0203\Delta(s)
\end{equation}
\begin{equation}
s\Theta(s) = 56.7Q(s)
\end{equation}
Μετά από μερικές απλές αλγεβρικές πράξεις καταλήγουμε στην ακόλουθη συνάρτηση μεταφοράς
\begin{equation}
P(s) = \frac{\Theta(s)}{\Delta(s)} = \frac{1.151s+0.1774}{s^3+0.739s^2+0.921s}
\label{eq:aircraft_tf}
\end{equation}

\subsection{Πείραμα}

Σε αυτό το παράδειγμα θέλουμε να δούμε αν ο αυτο--ρυθμιζόμενος PID ελεγκτής είναι σε θέση να ελέγξει ικανοποιητικά τη γωνία κλίσης του αεροσκάφους. Συγκεκριμένα, θέλουμε για βηματική είσοδο αναφοράς να μην παρουσιάζει μεγάλη υπερακόντιση, να μην έχει υπερβολικά μεγάλο χρόνο ανύψωσης και να παρουσιάζει πολύ μικρό σφάλμα μόνιμης κατάστασης. Στο πείραμα η είσοδος θα έχει τιμή $\delta=0.2$ rad (περίπου 11 μοίρες).

\subsubsection{Απόκριση Χωρίς Έλεγχο}

\begin{figure}[h]
  \centering
  \includegraphics[width=\textwidth,height=5cm,keepaspectratio]{aircraft_no_control}
  \caption{Απόκριση του αεροσκάφους σε είσοδο $\delta = 0.2$ rad χωρίς τη χρήση ελέγχου}
  \label{fig:aircraft_no_control}
\end{figure}

Όπως φαίνεται και από το παρακάτω σχήμα, το σύστημα όταν δρα χωρίς ελεγκτή δεν είναι ικανό να καλύψει τις απαιτήσεις ελέγχου. Ενώ η απόκριση του δεν παρουσιάζει υπερακόντιση, είναι σχετικά αργή και έχει ποσοστό σφάλματος μόνιμης κατάστασης περίπου $14\%$.

\subsubsection{Αναλογικός Έλεγχος}

Στο Σχήμα \ref{fig:aircraft_p} φαίνεται η απόκριση του συστήματος όταν σε αυτό εφαρμόζεται μόνο ο αναλογικός έλεγχος του PID ελεγκτή. Ο PID ελεγκτής έχει ρυθμιστεί στη λειτουργία ``Auto Tuning", συνεπώς μόλις ανιχνεύσει μόνιμο σφάλμα πάνω από $5\%$ για περισσότερο από πέντε δευτερόλεπτα εκτελεί αυτόματα το πείραμα relay. Δέκα ταλαντώσεις μετά, και ενώ το πλάτος τους είναι \emph{περίπου} σταθερό, ο αλγόριθμος διακόπτει το πείραμα και θέτει σε λειτουργία πάλι τον ελεγκτή χρησιμοποιώντας μόνο τον αναλογικό του όρο. Από τη γραφική παράσταση βλέπουμε ότι το μόνιμο σφάλμα είναι μικρότερο από $1\%$ αλλά η απόκριση παρουσιάζει υπερβολική υπερακόντιση και ταλαντώσεις.

\begin{figure}[h]
  \centering
  \includegraphics[width=\textwidth,height=5cm,keepaspectratio]{aircraft_p}
  \caption{Απόκριση του αεροσκάφους σε είσοδο $\delta = 0.2$ rad με τη χρήση αναλογικού ελέγχου}
  \label{fig:aircraft_p}
\end{figure}

\subsubsection{Αναλογικός -- Διαφορικός Έλεγχος}

Αφού το μόνιμο σφάλμα βρίσκεται σε ικανοποιητικά επίπεδα, η χρήση του ολοκληρωτικού όρου κρίνεται περιττή για το συγκεκριμένο σύστημα. Για τη βελτίωση της μεταβατικής κατάστασης εισάγουμε τον διαφορικό όρο. Η προσθήκη αυτή έχει εμφανή αποτελέσματα στην απόκριση του συστήματος όπως φαίνεται και στο παρακάτω σχήμα.

\begin{figure}[h]
  \centering
  \includegraphics[width=\textwidth,height=5cm,keepaspectratio]{aircraft_pd}
  \caption{Απόκριση του αεροσκάφους σε είσοδο $\delta = 0.2$ rad με τη χρήση αναλογικού -- διαφορικού ελέγχου}
  \label{fig:aircraft_pd}
\end{figure}

\subsubsection{Άλλες Αποκρίσεις}

\begin{figure}[h]
  \centering
  \includegraphics[width=\textwidth,height=5cm,keepaspectratio]{aircraft_pd_slow}
  \caption{Απόκριση του αεροσκάφους σε είσοδο $\delta = 0.2$ rad με τη χρήση αναλογικού -- διαφορικού ελέγχου και επιθυμητή ταχύτητα απόκρισης ``Slow"}
  \label{fig:aircraft_pd_slow}
\end{figure}

Εδώ φαίνονται οι αποκρίσεις του συστήματος για διαφορετικές ρυθμίσεις του προγράμματος. Συγκεκριμένα, στο Σχήμα \ref{fig:aircraft_pd_slow} φαίνεται η απόκριση του συστήματος όταν τα κέρδη έχουν υπολογιστεί με τους τύπους Ziegler -- Nichols και η επιθυμητή ταχύτητα έχει ρυθμιστεί στο ``Slow", ενώ στο Σχήμα \ref{fig:aircraft_pd_TL} φαίνεται η απόκριση όταν τα κέρδη έχουν υπολογιστεί με τους τύπους Tyreus -- Luyben.

Στην πρώτη περίπτωση το σύστημα δεν παρουσιάζει καθόλου υπερακόντιση κάτι που είναι θετικό αλλά όχι αναγκαίο. Στη δεύτερη περίπτωση η υπερακόντιση του συστήματος είναι μεγαλύτερη από αυτή στις προηγούμενες περιπτώσεις χωρίς ιδιαίτερη διαφορά στο χρόνο ανύψωσης.



\begin{figure}[h]
  \centering
  \includegraphics[width=\textwidth,height=5cm,keepaspectratio]{aircraft_pd_TL}
  \caption{Απόκριση του αεροσκάφους σε είσοδο $\delta = 0.2$ rad με τη χρήση αναλογικού -- διαφορικού ελέγχου που τα κέρδη έχουν υπολογιστεί με τους τύπους Tyreus -- Luyben}
  \label{fig:aircraft_pd_TL}
\end{figure}

\subsubsection{Αντιμετώπιση Διαταραχών}

Στο Σχήμα \ref{fig:aircraft_disturbances} φαίνεται η απόκριση του συστήματος υπό την παρουσία στιγμιαίων διαταραχών. Εύκολα γίνεται αντιληπτό ότι ο ελεγκτής παρουσιάζει πολύ καλή συμπεριφορά κάτω από συνθήκες διαταραχών αφού δεν αφήνει την έξοδο του συστήματος να αποκλίνει σχεδόν καθόλου από το επιθυμητό σημείο.

\begin{figure}[h]
  \centering
  \includegraphics[width=\textwidth,height=5cm,keepaspectratio]{aircraft_disturbances}
  \caption{Απόκριση του συστήματος στην παρουσία διαταραχών με χρήση αναλογικού -- διαφορικού ελέγχου}
  \label{fig:aircraft_disturbances}
\end{figure}

\subsection{Συμπεράσματα}

Σε αυτή την ενότητα το σύστημα που ήταν υπό έλεγχο ήταν αυτό ενός αεροσκάφους. Επειδή οι εξισώσεις που διέπουν τη δυναμική και την κίνηση ενός αεροσκάφους είναι περίπλοκες έγιναν κάποιοι συμβιβασμοί. Έτσι, καταλήξαμε σε μια γραμμική εκτίμηση του συστήματος που μας έδωσε τη συνάρτηση μεταφοράς που φαίνεται στην εξίσωση \ref{eq:aircraft_tf}. Ως στόχοι του ελέγχου τέθηκαν η μικρή υπερακόντιση, ο γρήγορος χρόνος ανύψωσης και το μικρό σφάλμα μόνιμης κατάστασης.

Το πρώτο πράγμα που ελέγχθηκε ήταν η απόκριση του συστήματος χωρίς τη χρήση ελεγκτή. Αυτή γρήγορα κρίθηκε μη ικανοποιητική καθώς δεν πληρούσε κανένα από τα κριτήρια ελέγχου. Στη συνέχεια, χρησιμοποιήθηκε ο αναλογικός όρος του ελεγκτή. Παρόλο που αυτός βελτίωσε αισθητά το σφάλμα μόνιμης κατάστασης, χειροτέρευσε τα μεταβατικά φαινόμενα του συστήματος και οδήγησε σε υψηλό ποσοστό υπερακόντισης και σε εκτεταμένες ταλαντώσεις. Επειδή το σύστημα έχει από μόνο του έναν πόλο στο μηδέν, το σφάλμα μόνιμης κατάστασης θα τείνει να μηδενιστεί χωρίς τη χρήση ολοκληρωτικού όρου. Συνεπώς, προστέθηκε ο διαφορικός όρος στον ελεγκτή για τη βελτίωση της ευστάθειας του αεροσκάφους. Με τη χρήση αναλογικού -- διαφορικού ελέγχου οι ταλαντώσεις εξαφανίστηκαν, η υπερακόντιση έγινε πολύ μικρή και το σφάλμα συγκλίνει πιο γρήγορα στο μηδέν ενώ και οι χρόνοι ανύψωσης και ηρεμίας μειώθηκαν αισθητά. Τέλος, ο έλεγχος παρουσιάζει πολύ καλή συμπεριφορά υπό τη συνθήκη διαταραχών. Συνοψίζοντας, ο αυτο--ρυθμιζόμενος PID ελεγκτής κατάφερε να ελέγξει πολύ ικανοποιητικά το σύστημα του αεροσκάφους.\newpage

\section{Σύστημα Beam \& Ball} \label{sec:ball_beam}

\subsection{Εισαγωγή}

Το τελευταίο σύστημα που θα προσπαθήσουμε να ελέγξουμε είναι το σύστημα Δοκός -- Σφαίρα όπως αυτό φαίνεται στο Σχήμα \ref{fig:bb2}. Μία μπάλα τοποθετείται σε μία δοκό όπου επιτρέπεται να κυλήσει με ένα βαθμό ελευθερίας κατά μήκος της δοκού. Ένας μοχλοβραχίονας είναι στερεωμένος στο ένα άκρο της δοκού και ένα σερβο γρανάζι στο άλλο. Καθώς ο σερβομηχανισμός γυρίζει σε γωνία $\theta$, ο μοχλός αλλάζει τη γωνία της δέσμης κατά $\alpha$. Όταν αλλάζει η γωνία από την οριζόντια θέση, η βαρύτητα προκαλεί την κύλιση της σφαίρας κατά μήκος της δέσμης. Ένας ελεγκτής θα σχεδιαστεί για αυτό το σύστημα έτσι ώστε να μπορεί να χειραγωγηθεί η θέση της μπάλας.

\begin{figure}[h]
  \centering
  \includegraphics[width=\textwidth,height=5cm,keepaspectratio]{bb2}
  \caption{Μοντέλο συστήματος Δοκός -- Σφαίρα}
  \label{fig:bb2}
\end{figure}

\subsection{Μαθηματικό Μοντέλο} \label{subsec:ball_beam_math_model}

Η δεύτερη παράγωγος της γωνίας εισόδου $\alpha$ επηρεάζει στην πραγματικότητα τη δεύτερο παράγωγο του $r$. Ωστόσο, θα αγνοήσουμε αυτή τη συμβολή. Η Lagrangian εξίσωση κίνησης για την μπάλα δίνεται από την ακόλουθη εξίσωση
\begin{equation}
0 = \left(\frac{J}{R^2}+m\right)\ddot{r}+mg\sin\alpha - mr\dot{\alpha}^2
\label{eq:ball_beam_non_linear}
\end{equation}
Η παραπάνω εξίσωση είναι μη γραμμική. Σε αυτή την ενότητα θα επιχειρήσουμε να ελέγξουμε, πέρα από τη γραμμική προσέγγιση του συστήματος αυτού, και το μη γραμμικό σύστημα. Αρχικά, θα συνεχίσουμε με τη γραμμική προσέγγιση. Μετά την προσπάθεια ελέγχου του γραμμικού συστήματος θα ασχοληθούμε και με το μη γραμμικό μοντέλο του.

\subsection{Γραμμικό Μοντέλο}

Γραμμικοποίηση αυτής της εξίσωσης γύρω από τη γωνία της ράβδου, $\alpha = 0$, μας δίνει την ακόλουθη γραμμική προσέγγιση του συστήματος:
\begin{equation}
\left(\frac{J}{R^2}+m\right)\ddot{r} = -mg\alpha
\end{equation}
Η εξίσωση που συνδέει τη γωνία της ράβδου με τη γωνία του γραναζιού μπορεί να προσεγγιστεί ως γραμμική από την παρακάτω εξίσωση:
\begin{equation}
\alpha = \frac{d}{L}\theta
\label{eq:alpha_to_theta}
\end{equation}
Αντικαθιστώντας αυτό στην προηγούμενη εξίσωση, παίρνουμε:
\begin{equation}
\left(\frac{J}{R^2}+m\right)\ddot{r} = -mg\frac{d}{L}\theta
\end{equation}
Λαμβάνοντας το μετασχηματισμό Laplace της παραπάνω εξίσωσης, βρίσκουμε την ακόλουθη εξίσωση:
\begin{equation}
\left(\frac{J}{R^2}+m\right)R(s)s^2 = -mg\frac{d}{L}\Theta(s)
\end{equation}
Αναδιατάσσοντας την εξίσωση αυτή βρίσκουμε τη συνάρτηση μεταφοράς από τη γωνία του γραναζιού $\Theta(s)$ στη θέση της μπάλας $R(s)$.
\begin{equation}
P(s) = \frac{R(s)}{\Theta(s)} = -\frac{mgd}{L\left(\frac{J}{R^2}+m\right)}\frac{1}{s^2} \left[\frac{m}{rad}\right]
\end{equation} 
Όπως φαίνεται από τη συνάρτηση μεταφοράς, το παραπάνω μοντέλο έχει διπλό πόλο στο μηδέν οπότε παρουσιάζει αρκετή δυσκολία στον έλεγχό του.

\subsubsection{Πείραμα}

Στο συγκεκριμένο παράδειγμα κάνουμε τις παραδοχές ότι η μπάλα εκτελεί κύλιση χωρίς ολίσθηση κατά μήκος της ράβδου και επίσης ότι η τριβή μεταξύ της μπάλας και της δοκού είναι αμελητέα. Οι παράμετροι του συστήματος δίνονται ως εξής:
\begin{flushleft}
\begin{tabular}{lll}
$\mathbf{m}$ & μάζα της μπάλας & $0.11\ kg$ \\  
$\mathbf{R}$ & ακτίνα της μπάλας & $0.015\ m$ \\ 
$\mathbf{d}$ & απόσταση του μοχλοβραχίονα από το κέντρο του γραναζιού & $0.03\ m$ \\ 
$\mathbf{g}$ & επιτάχυνση της βαρύτητας & $9,8\ \frac{m}{s^2}$ \\ 
$\mathbf{L}$ & μήκος της δοκού & $1\ m$ \\ 
$\mathbf{J}$ & ροπή αδράνειας της μπάλας & $ 9.99*10^{-6}\ kgm^2 $ \\ 
\end{tabular}
\end{flushleft}

\paragraph{Απόκριση Χωρίς Έλεγχο}\hfill

Αρχικά βλέπουμε πώς το σύστημα αποκρίνεται όταν σε αυτό δεν εφαρμόζεται έλεγχος και εισάγουμε μια βηματική είσοδο με πλάτος $\text{setpoint}=0.25\ m$. Αυτό σημαίνει ότι θέλουμε η μπάλα να σταθεροποιηθεί $25\ \text{εκατοστά}$ μακριά από την άκρη της δοκού ή αλλιώς στο $1/4$ του μήκους της. Από το Σχήμα \ref{fig:ball_beam_no_control} εύκολα συμπεραίνουμε ότι το σύστημα είναι οριακά ευσταθές και ότι απαιτείται έλεγχος.

\begin{figure}[h]
  \centering
  \includegraphics[width=\textwidth,height=5cm,keepaspectratio]{ball_beam_no_control}
  \caption{Βηματική απόκριση του συστήματος δοκού -- σφαίρας χωρίς έλεγχο}
  \label{fig:ball_beam_no_control}
\end{figure}

\paragraph{Αναλογικός Έλεγχος}\hfill

\begin{figure}[h]
  \centering
  \includegraphics[width=\textwidth,height=5cm,keepaspectratio]{ball_beam_proportional}
  \caption{Βηματική απόκριση του συστήματος δοκού -- μπάλας με χρήση αναλογικού ελέγχου}
  \label{fig:ball_beam_proportional}
\end{figure}

Το Σχήμα \ref{fig:ball_beam_proportional} δείχνει την απόκριση του συστήματος με τη χρήση του αναλογικού όρου του οποίου το κέρδος έχει υπολογιστεί αυτόματα. Βλέπουμε ότι ο αναλογικός όρος όχι απλά δε βελτίωσε την απόκριση αλλά οδηγεί το σύστημα να εκτελεί ταλαντώσεις ολοένα και αυξανόμενου πλάτους. Επίσης, κατά τη διάρκεια του πειράματος relay το σύστημα δεν εισέρχεται σε κύκλο αμείωτων ταλαντώσεων. Αντιθέτως, οι ταλαντώσεις του έχουν πλάτος που αυξάνεται με την πάροδο του χρόνου. Έχει ενδιαφέρον να δούμε την απόδοση του αυτο--ρυθμιζόμενου PID ελεγκτή υπό αυτές τις συνθήκες για ένα απαιτητικό σύστημα σαν αυτό.



\paragraph{Αναλογικός -- Διαφορικός Έλεγχος}\hfill

Για τη βελτίωση της συμπεριφοράς του συστήματος εισάγουμε και το διαφορικό όρο στον ελεγκτή. Όπως φαίνεται στο Σχήμα \ref{fig:ball_beam_pd} το σύστημα με την προσθήκη αυτή έγινε πλέον ευσταθές. Επίσης παρουσιάζει μικρό ποσοστό υπερακόντισης και τείνει γρήγορα στο μηδέν.

\begin{figure}[h]
  \centering
  \includegraphics[width=\textwidth,height=5cm,keepaspectratio]{ball_beam_pd}
  \caption{Βηματική απόκριση του συστήματος δοκού -- μπάλας με χρήση αναλογικού -- διαφορικού ελέγχου}
  \label{fig:ball_beam_pd}
\end{figure}

\begin{figure}[h]
  \centering
  \includegraphics[width=\textwidth,height=5cm,keepaspectratio]{ball_beam_pd_slow}
  \caption{Βηματική απόκριση του συστήματος δοκού -- μπάλας με χρήση αναλογικού -- διαφορικού ελέγχου και επιθυμητή ταχύτητα απόκρισης ``Slow"}
  \label{fig:ball_beam_pd_slow}
\end{figure}

Επίσης στο Σχήμα \ref{fig:ball_beam_pd_slow} φαίνεται πώς το σύστημα αποκρίνεται στην ίδια είσοδο αν η ταχύτητά του έχει οριστεί σε ``Slow". Οι χρόνοι ανύψωσης και ηρεμίας έχουν αυξηθεί αισθητά αλλά το σύστημα πλέον παρουσιάζει μηδενική υπερακόντιση.

\paragraph{Αντιμετώπιση Διαταραχών} \label{par:disturbances}\hfill

\begin{figure}[h]
  \centering
  \includegraphics[width=\textwidth,height=5cm,keepaspectratio]{ball_beam_disturbances}
  \caption{Απόκριση του συστήματος στην παρουσία διαταραχών με χρήση αναλογικού -- διαφορικού ελέγχου}
  \label{fig:ball_beam_disturbances}
\end{figure}

Στο Σχήμα \ref{fig:ball_beam_disturbances} βλέπουμε πώς συμπεριφέρεται το σύστημα δοκός -- μπάλα υπό την παρουσία διαταραχών. Εύκολα γίνεται αντιληπτό ότι ο ελεγκτής παρουσιάζει πολύ καλή συμπεριφορά, αφού δεν αφήνει τη μπάλα να κυλίσει μακριά από το επιθυμητό σημείο και επίσης την επιστρέφει σχεδόν αμέσως στη θέση ισορροπίας της.

\subsection{Μη Γραμμικό Μοντέλο}

Όπως είδαμε στην υποενότητα \ref{subsec:ball_beam_math_model}, η εξίσωση κίνησης της σφαίρας πάνω στη δοκό δίνεται από τη μη γραμμική διαφορική εξίσωση:
\begin{equation*}
0 = \left(\frac{J}{R^2}+m\right)\ddot{r}+mg\sin\alpha - mr\dot{\alpha}^2
\end{equation*}
Επίσης, έχουμε ότι η γωνία $\alpha$ της δοκού συνδέεται με τη γωνία $\theta$ του γραναζιού μέσω του τύπου \ref{eq:alpha_to_theta}:
\begin{equation*}
\alpha = \frac{d}{L}\theta
\end{equation*}

Είναι γνωστό ότι οι συναρτήσεις μεταφοράς χρησιμοποιούνται για να περιγράψουν γραμμικά, χρονικώς αμετάβλητα συστήματα (\emph{LTI Systems}). Συνεπώς, για την περιγραφή του μη γραμμικού μοντέλου αυτού του συστήματος δεν είναι δυνατή η χρήση συνάρτησης μεταφοράς όπως έγινε στα προηγούμενα παραδείγματα.

Το ``Simulation Loop" του LabVIEW επιτρέπει την απευθείας μοντελοποίηση της μη γραμμικής διαφορικής εξίσωσης (\ref{eq:ball_beam_non_linear}). Αυτό το μοντέλο φαίνεται στο Σχήμα \ref{fig:Non-Lineard}.

\begin{figure}[h]
  \centering
  \includegraphics[width=\textwidth,height=5cm,keepaspectratio]{Non-Lineard}
  \caption{Δομικό διάγραμμα του μη γραμμικού μοντέλου του συστήματος δοκός -- μπάλα}
  \label{fig:Non-Lineard}
\end{figure}

Το μοντέλο δέχεται ως είσοδο ``input" τη γωνία $\theta$ και ως έξοδο ``output" αποδίδει την απόσταση $r$ της σφαίρας από την αρχή της δοκού. Επίσης, θεωρούμε ότι οι παράμετροι του συστήματος έχουν τις ίδιες τιμές με πριν. Στόχος μας είναι να δούμε κατά πόσο είναι εφικτό ο αυτο--ρυθμιζόμενος PID ελεγκτής να υπολογίσει κέρδη που θα καταφέρουν να ελέγξουν αποδοτικά το μη γραμμικό μοντέλο του συστήματος αυτού. Ως επιθυμητή θέση της μπάλας (\emph{setpoint}) είναι, όπως και πριν, τα $25$ εκατοστά από το άκρο της δοκού.

\paragraph{Απόκριση Χωρίς Έλεγχο}\hfill

\begin{figure}[H]
  \centering
  \includegraphics[width=\textwidth,height=5cm,keepaspectratio]{ball_beam_non_linear_no_control}
  \caption{Βηματική απόκριση του μη γραμμικού μοντέλου του συστήματος δοκός -- μπάλα χωρίς την εφαρμογή ελέγχου}
  \label{fig:ball_beam_non_linear_no_control}
\end{figure}

Όπως κάθε πείραμα, έτσι και αυτό ξεκινάει με τη βηματική απόκριση του συστήματος όταν δεν εφαρμόζεται έλεγχος. Από το παραπάνω σχήμα φαίνεται ότι η απόκρισή του μη γραμμικού μοντέλου είναι παρόμοια με την απόκριση της γραμμικής προσέγγισης του που εξετάσαμε στην προηγούμενη υποενότητα. Συγκεκριμένα, το σύστημα είναι οριακά ευσταθές αφού εκτελεί αμείωτες ταλαντώσεις.\newpage

\paragraph{Αναλογικός Έλεγχος}\hfill

\begin{figure}[h]
  \centering
  \includegraphics[width=\textwidth,height=5cm,keepaspectratio]{ball_beam_non_linear_p}
  \caption{Βηματική απόκριση του μη γραμμικού μοντέλου του συστήματος δοκός -- μπάλα με την εφαρμογή αναλογικού ελέγχου}
  \label{fig:ball_beam_non_linear_p}
\end{figure}

Από το Σχήμα \ref{fig:ball_beam_non_linear_p} φαίνεται ότι η εφαρμογή αναλογικού ελέγχου στο μη γραμμικό σύστημα έχει την ίδια επίδραση με αυτή που είχε ο αναλογικός έλεγχος στη γραμμική προσέγγιση. Το σύστημα συνεχίζει να εκτελεί ταλαντώσεις αλλά αυτή τη φορά θα μπορούσαμε να πούμε ότι είναι ασταθές αφού το πλάτος των ταλαντώσεων αυξάνεται αργά αλλά σταθερά με την πάροδο του χρόνου.

\paragraph{Αναλογικός -- Διαφορικός Έλεγχος}\hfill

\begin{figure}[H]
  \centering
  \includegraphics[width=\textwidth,height=5cm,keepaspectratio]{ball_beam_non_linear_pd_normal}
  \caption{Βηματική απόκριση του μη γραμμικού μοντέλου του συστήματος δοκός -- μπάλα με την εφαρμογή αναλογικού -- διαφορικού ελέγχου}
  \label{fig:ball_beam_non_linear_pd_normal}
\end{figure}

\begin{figure}[h]
  \centering
  \includegraphics[width=\textwidth,height=5cm,keepaspectratio]{ball_beam_non_linear_pd}
  \caption{Βηματική απόκριση του μη γραμμικού μοντέλου του συστήματος δοκός -- μπάλα με την εφαρμογή αναλογικού -- διαφορικού ελέγχου και επιθυμητή ταχύτητα απόκρισης ``Slow"}
  \label{fig:ball_beam_non_linear_pd}
\end{figure}

Στα σχήματα \ref{fig:ball_beam_non_linear_pd_normal} και \ref{fig:ball_beam_non_linear_pd} φαίνονται οι αποκρίσεις του συστήματος όταν σε αυτό εφαρμόζεται αναλογικός -- διαφορικός έλεγχος για ταχύτητα απόκρισης ``Normal" και ``Slow" αντίστοιχα. Από τις γραφικές παραστάσεις φαίνεται ότι ο αυτο--ρυθμιζόμενος PID ελεγκτής κάνει αρκετά καλή δουλειά στον έλεγχο του συστήματος. Η απόκριση παρουσιάζει μηδενικό σφάλμα μόνιμης κατάστασης και μικρό ποσοστό υπερακόντισης. Στη λειτουργία ``Slow" ο έλεγχος ανταλλάζει τον μικρό χρόνο ανύψωσης για να πετύχει μηδενικό ποσοστό υπερακόντισης.

\subsection{Γραμμικό και Μη Γραμμικό Μοντέλο -- Ομοιότητες και Διαφορές}

Εκ πρώτης όψεως, τα αποτελέσματα των προηγούμενων παραγράφων υποδεικνύουν ότι ο αυτο--ρυθμιζόμενος PID ελεγκτής είναι ικανός να ελέγξει το ίδιο αποδοτικά τόσο το γραμμικό μοντέλο όσο και το μη γραμμικό. Όμως, με μία πιο προσεκτική ματιά γίνεται εύκολα αντιληπτό ότι ο έλεγχος στη μη γραμμική περίπτωση δεν είναι ικανοποιητικός.

Αρχικά, κατά τη διάρκεια του πειράματος relay, η αντίδραση του μη γραμμικού μοντέλου είναι πιο επιθετική, οδηγώντας την έξοδο του συστήματος να εκτελεί ταλαντώσεις μεγαλύτερου πλάτους πιο νωρίς, σε σχέση με το γραμμικό μοντέλο.

Η πραγματική διαφορά όμως φαίνεται όταν επιχειρείται να γίνει αλλαγή του επιθυμητού σημείου. Ας πούμε για παράδειγμα ότι η μπάλα έχει ισορροπήσει στα $25$ εκατοστά χρησιμοποιώντας αναλογικό -- διαφορικό έλεγχο και θέλουμε τώρα να ισορροπήσει στα $50$ εκατοστά. Τέτοιες αλλαγές είναι πιθανόν να γίνονται συνέχεια στα πλαίσια λειτουργίας του συστήματος. Στα επόμενα σχήματα φαίνεται πώς αποκρίνεται το γραμμικό (Σχήμα \ref{fig:bb_linear_setpoint_change}) και πώς το μη γραμμικό μοντέλο (Σχήμα \ref{fig:bb_non_linear_setpoint_change_unstable}).

\begin{figure}[h]
  \centering
  \includegraphics[width=\textwidth,height=5cm,keepaspectratio]{bb_linear_setpoint_change}
  \caption{Απόκριση του γραμμικού μοντέλου σε διάφορες αλλαγές του επιθυμητού σημείου}
  \label{fig:bb_linear_setpoint_change}
\end{figure}

\begin{figure}[h]
  \centering
  \includegraphics[width=\textwidth,height=5cm,keepaspectratio]{bb_non_linear_setpoint_change_unstable}
  \caption{Απόκριση του μη γραμμικού μοντέλου σε διάφορες αλλαγές του επιθυμητού σημείου}
  \label{fig:bb_non_linear_setpoint_change_unstable}
\end{figure}

Από τις δύο γραφικές παραστάσεις είναι εύκολο να αντιληφθούμε τι πάει λάθος με τον έλεγχο του μη γραμμικού μοντέλου. Στη γραμμική περίπτωση, όσες φορές και να αλλάξει το επιθυμητό σημείο, το σύστημα παραμένει ευσταθές και η απόδοση του ελέγχου δεν αλλάζει. Σε κάθε αλλαγή το σύστημα έχει αναμενόμενη συμπεριφορά και παρόμοια μεταβατική και μόνιμη κατάσταση.\newpage

Αντιθέτως, η απόκριση του μη γραμμικού συστήματος δεν παρουσιάζει παρόμοια συμπεριφορά σε κάθε αλλαγή. Όταν η τιμή του επιθυμητού σημείου μεταβάλλεται από $0.5$ σε $0.3$ η έξοδος του συστήματος πρώτα αυξάνεται αντί να μειωθεί. Επίσης, όταν επιχειρείται να επιστρέψει στα $50$ εκατοστά βλέπουμε ότι το σύστημα γίνεται ασταθές και η έξοδός του γρήγορα τείνει προς το άπειρο.

Παρόμοια φαινόμενα παρατηρούνται και όταν στο σύστημα εμφανίζονται διαταραχές. Στο Σχήμα \ref{fig:bb_linear_disturbances} φαίνεται πώς ο ελεγκτής αντιμετωπίζει τις διαταραχές στο γραμμικό σύστημα, ενώ στο Σχήμα \ref{fig:bb_non_linear_disturbances} φαίνεται πώς ο ελεγκτής αντιμετωπίζει τις διαταραχές στο μη γραμμικό σύστημα.

Σε πλήρη αντιστοιχία με πριν βλέπουμε ότι το σύστημα που βασίζεται στη γραμμική προσέγγιση έχει αναμενόμενη συμπεριφορά σε κάθε διαταραχή, όπως αυτή περιγράφηκε στο τμήμα \ref{par:disturbances}. Το ίδιο δεν μπορεί να ειπωθεί και για το μη γραμμικό σύστημα αφού, παρόλο που ο ελεγκτής έχει καταφέρει σε πρώτη φάση να το φέρει στην επιθυμητή τιμή, υπό την παρουσία των διαταραχών η απόκρισή του δεν αργεί να χάσει την ευστάθειά της και να πάρει τεράστιες τιμές.

\begin{figure}[h]
  \centering
  \includegraphics[width=\textwidth,height=5cm,keepaspectratio]{bb_linear_disturbances}
  \caption{Απόκριση του γραμμικού συστήματος υπό την παρουσία διαταραχών}
  \label{fig:bb_linear_disturbances}
\end{figure}

\begin{figure}[h]
  \centering
  \includegraphics[width=\textwidth,height=5cm,keepaspectratio]{bb_non_linear_disturbances}
  \caption{Απόκριση του μη γραμμικού συστήματος υπό την παρουσία διαταραχών}
  \label{fig:bb_non_linear_disturbances}
\end{figure}

\subsection{Συμπεράσματα}

Το τελευταίο σύστημα που επιχειρήσαμε να ελέγξουμε χρησιμοποιώντας τον αυτο--ρυθμιζόμενο PID ελεγκτή είναι το σύστημα δοκού -- μπάλας. Σε αντίθεση με το παράδειγμα του ανεστραμμένου εκκρεμούς ή του αεροσκάφους, στα οποία χρησιμοποιούσαμε μόνο τη γραμμική προσέγγιση τους, σε αυτό το παράδειγμα προσπαθήσαμε να ελέγξουμε τόσο το γραμμικό όσο και το μη γραμμικό μοντέλο του συστήματος.

Αυτό που έγινε γρήγορα αντιληπτό ήταν ότι τόσο το γραμμικό όσο και το μη γραμμικό σύστημα δεν εκτελούσαν σταθερές ταλαντώσεις κατά τη διάρκεια του πειράματος relay. Αν ωστόσο, το πείραμα διακοπτόταν πριν το πλάτος των ταλαντώσεων γίνει πολύ μεγάλο ή το σύστημα γίνει ασταθές, οι τιμές των κερδών που υπολόγιζε ο αλγόριθμος ήταν αρκετές για να παρέχουν έλεγχο.

Όσον αφορά το \textbf{γραμμικό μοντέλο}, επειδή η συνάρτηση μεταφοράς περιέχει από μόνη της δύο πόλους στο μηδέν, δεν χρειάζεται ο ολοκληρωτικός όρος για να εξαλείψει το μόνιμο σφάλμα. Συνεπώς, η χρήση του αναλογικού και του διαφορικού όρου κρίθηκε επαρκής για τον έλεγχο της θέσης της μπάλας. Το κλειστό σύστημα παρουσίαζε μικρή υπερακόντιση και ικανοποιητικούς χρόνους ανύψωσης και ηρεμίας. Επίσης, άμα ο ελεγκτής ρυθμιζόταν στην ``αργή" λειτουργία η υπερακόντιση ήταν μηδενική με αντάλλαγμα μια μείωση στην ταχύτητα του συστήματος.

Όσον αφορά το \textbf{μη γραμμικό μοντέλο}, ο ελεγκτής όπως ήταν αναμενόμενο δεν κατάφερε να παρέχει ικανοποιητικό έλεγχο. Ενώ φαινομενικά στην αρχή έπιασε τους στόχους ελέγχου, η συμπεριφορά του κατά τη διάρκεια του πειράματος ήταν απρόβλεπτη και μη επιθυμητή. Η παρουσία διαταραχών ή η αλλαγή του επιθυμητού σημείου εύκολα μπορούσαν να οδηγήσουν το σύστημα στην αστάθεια και την έξοδό του να πάρει πολύ μεγάλες τιμές. Ακόμα και κατά τη διάρκεια του πειράματος relay, οι ταλαντώσεις που εκτελούσε το σύστημα αποκτούσαν γρηγορότερα μεγαλύτερο πλάτος. Συνοψίζοντας, ο έλεγχος που παρέχει ο αυτο--ρυθμιζόμενος PID ελεγκτής δεν καταφέρνει να αντιμετωπίσει την απρόβλεπτη συμπεριφορά που επιδεικνύει το μη γραμμικό μοντέλο του συστήματος δοκός -- μπάλα και ως αποτέλεσμα κρίνεται ανεπαρκές. \newpage

\section{Σύνοψη}

Στο κεφάλαιο αυτό έγινε αναλυτική παρουσίαση των προσομοιώσεων του αυτο--ρυθμιζόμενου PID ελεγκτή που αναπτύχθηκε στα πλαίσια αυτής της διπλωματικής εργασίας. Κάθε πρόβλημα ελέγχου αντιμετωπίστηκε ξεχωριστά. Τα συστήματα που επιλέχθηκαν αποτελούν κλασικά προβλήματα αυτομάτου ελέγχου που περιλαμβάνονται σε πολλά εκπαιδευτικά βιβλία και εγχειρίδια.

Όπως ήταν αναμενόμενο, κάθε σύστημα αντιδράει διαφορετικά στη διαδικασία ρύθμισης του ελεγκτή. Για παράδειγμα, τα συστήματα πρώτης τάξης (Ενότητα \ref{sec:cruise_control}) εκτελούν ταλαντώσεις πολύ μικρού πλάτους και περιόδου ενώ τα συστήματα δεύτερης τάξης (Ενότητα \ref{sec:mass_spring_damper} και Ενότητα \ref{sec:motor_speed}) εκτελούν ταλαντώσεις περίπου πέντε με έξι φορές μικρότερου πλάτους αλλά περίπου οχτώ φορές μεγαλύτερης περιόδου. 

Επίσης, ανάλογα με τις ρυθμίσεις του ελεγκτή, η απόκριση του κλειστού συστήματος ελέγχου για κάθε σύστημα διαφέρει. Από τις προσομοιώσεις φάνηκε ότι αν η επιθυμητή ταχύτητα απόκρισης του συστήματος τεθεί στο ``Slow" τότε η μεταβολή της εξόδου του συστήματος είναι πιο ήπια από την άποψη ότι δεν παρουσιάζει υπερακόντιση αλλά χρειάζεται περισσότερο χρόνο να φτάσει στην επιθυμητή τιμή και να ηρεμήσει. Αυτό συνάδει με τη θεωρία που αναφέρθηκε στην Υποενότητα \ref{subsce:rise_time}. 

Ακόμα, ανάλογα με το ποιοι τύποι χρησιμοποιήθηκαν για τον υπολογισμό των κερδών του ελεγκτή έχουμε διαφορετικά αποτελέσματα. Οι Ziegler -- Nichols αποκρίσεις παρουσίαζαν μια σχεδόν ομογενής απόσβεση των ταλαντώσεων της μετρούμενης μεταβλητής, ενώ οι Tyreus -- Luyben αποκρίσεις παρουσίαζαν μεγάλη πρώτη υπερακόντιση και σχεδόν καθόλου στη συνέχεια. Οπότε, οι ρυθμίσεις που επιλέγει ο χρήστης εξαρτώνται από τις απαιτήσεις του κάθε ελέγχου. Αν η εφαρμογή μπορεί να αντέξει κάποιο ποσοστό υπερακόντισης, τότε οι τύποι Ziegler -- Nichols στη ρύθμιση ``Normal" ή ``Fast" ή οι τύποι Tyreus -- Luyben μπορεί να είναι οι καταλληλότεροι. Αντιθέτως, αν η διεργασία δεν πρέπει να παρουσιάζει καθόλου υπερακόντιση τότε οι τύποι Ziegler -- Nichols στη ρύθμιση ``Slow" θα προσφέρουν μια πιο ικανοποιητική λύση ελέγχου.

Επιπλέον, κάτι που αξίζει να αναφερθεί είναι ότι ακόμα και σε συστήματα που είναι δύσκολο να ελεγχθούν ή που είναι ιδιαίτερα ευαίσθητα και εύκολο να οδηγηθούν στην αστάθεια, ο αλγόριθμος για τη ρύθμιση των κερδών του ελεγκτή απέδωσε αρκετά καλά αποτελέσματα. Οι τιμές που υπολογίστηκαν για τους όρους του PID ελεγκτή, ακόμα και όταν το σύστημα δεν εκτελούσε σταθερές ταλαντώσεις ή η διαδικασία ρύθμισης διακοπτόταν πρόωρα ήταν αρκετές για να αποδώσουν ικανοποιητικό έλεγχο.

Τέλος, όσον αφορά στο πρόγραμμα που υλοποιήθηκε στα πλαίσια αυτής της εργασίας και στη λειτουργία του αυτο--ρυθμιζόμενου PID ελεγκτή, είναι απαραίτητο να γίνουν δύο παρατηρήσεις. Η πρώτη έχει να κάνει με το ότι σε κανένα από τα παραδείγματα δεν τροποποιήθηκαν χειροκίνητα τα κέρδη που υπολογίστηκαν. Αυτό σημαίνει ότι οι αποκρίσεις που δείχνονται παραπάνω δεν είναι οι βέλτιστες που μπορεί να αποδώσει ένας ελεγκτής. Σε πραγματικές συνθήκες, οι τιμές αυτές οι οποίες υπολογίζονται αυτόματα, αποτελούν μια καλή αφετηρία για την περαιτέρω ρύθμιση του PID ελεγκτή. Το βασικό πλεονέκτημα του αυτο--ρυθμιζόμενου PID ελεγκτή είναι ότι μειώνει αρκετά τις προσπάθειες για τη ρύθμιση των κερδών του, αφού δίνει μια πρώτη καλή προσέγγιση αυτών χωρίς να ταλαιπωρεί πολύ τον εξοπλισμό. Οι αποκρίσεις λοιπόν που δείχνονται στις παραπάνω ενότητες, έχουν ακόμα χώρο για βελτίωση, με μικρή τροποποίηση των κερδών των όρων του ελεγκτή.

Η δεύτερη παρατήρηση έχει να κάνει με την εφαρμογή του προγράμματος σε πραγματικές εφαρμογές. Σε μία πραγματική εφαρμογή, ο αυτο--ρυθμιζόμενος PID ελεγκτής που έχει υλοποιηθεί σε αυτή την εργασία, μπορεί να μην είναι απευθείας εφαρμόσιμος με τον τρόπο που έχει φτιαχτεί. Για παράδειγμα, στο σύστημα με τη δοκό και τη μπάλα, σαν αρχή μέτρησης της απόστασης της μπάλας έχει θεωρηθεί η δεξιά άκρη της δοκού. Συνεπώς, κατά τη διάρκεια της αυτόματης ρύθμισης των κερδών, στην οποία η έξοδος του συστήματος ταλαντώνεται γύρω από το σημείο μηδέν, η μπάλα θα έπεφτε από τη δοκό όταν η έξοδος του συστήματος θα γινόταν αρνητική. Αυτό όμως δεν αποτρέπει τη χρήση του συγκεκριμένου προγράμματος ως μέσο προσομοίωσης για την αυτόματη εύρεση της αρχικής τιμής των κερδών ενός πραγματικού PID ελεγκτή και στη συνέχεια, την εφαρμογή τους σε πραγματικές συνθήκες.