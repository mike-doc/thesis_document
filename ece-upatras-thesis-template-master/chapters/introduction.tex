%!TEX root = ../main.tex

\chapter*{Εισαγωγή}
\markboth{Εισαγωγη}{}
%\vspace{-1.3in}
\lettrine[findent=2pt]{\fbox{\textbf{Η}}}{} εργασία αυτή έχει γίνει προσπάθεια να γραφεί σε ανεξάρτητα κεφάλαια, τα οποία θα δώσουν στον αναγνώστη τις απαιτούμενες γνώσεις ώστε να καταλάβει σε βάθος τις τεχνικές που χρησιμοποιούνται. Σε κάθε κεφάλαιο γίνεται αναλυτική παρουσίαση των τεχνικών καθώς και του υπόβαθρου που πρέπει να έχει κάποιος ώστε τις κατανοήσει, ωστόσο θεωρείται πως ο αναγνώστης έχει ήδη κάποιες γνώσεις στο χώρο του αυτομάτου ελέγχου και στην ανάλυση συστημάτων. Έτσι, βασικές έννοιες και μηχανισμοί της ανωτέρω περιοχής θα θεωρούνται δεδομένοι και δε θα γίνει κάποια ανάλυσή τους στο κείμενο αυτό, εκτός αν κρίνεται απαραίτητο.

Στην ενότητα αυτή θα δούμε μια σύντομη επισκόπηση των κεφαλαίων και της δομής της εργασίας αυτής. Στο Κεφάλαιο \ref{ch:chap1}, γίνεται μια αναφορά στο περιβάλλον LabVIEW, καθώς δεν αποτελεί μια κλασική, δομική γλώσσα προγραμματισμού.

Το Κεφάλαιο \ref{ch:chap2}, ασχολείται με τους PID ελεγκτές, την ιστορική τους εξέλιξη καθώς και τις αρχές που διέπουν τη λειτουργία τους.

Στο Κεφάλαιο \ref{ch:chap3}, αναλύονται με λεπτομέρεια οι τεχνικές ρύθμισης που χρησιμοποιήθηκαν για την υλοποίηση του συγκεκριμένου ελεγκτή. Επίσης, αναλύονται τα προβλήματα και οι περιορισμοί που καθιστούν τη ρύθμιση ενός, κατά τα άλλα, απλού ελεγκτή, δύσκολη υπόθεση. 

Στο Κεφάλαιο \ref{ch:chap4}, παρουσιάζεται η θεωρία της ``Relay" μεθόδου που χρησιμοποιείται για την αυτόματη ρύθμιση του ελεγκτή καθώς και το πρόγραμμα που αποτελεί τον PID ελεγκτή. Γίνεται εκτενής χρήση εικόνων, τόσο του εποπτικού πάνελ όσο και του λειτουργικού διαγράμματος κάθε συνάρτησης που υλοποιεί διαφορετική λειτουργία. Επίσης, σχολιάζεται η λογική πίσω από κάθε σχεδιαστική επιλογή και οι λόγοι που οδήγησαν σε αυτή.

Στο Κεφάλαιο \ref{ch:chap5}, παρουσιάζονται τα συστήματα που έχουν επιλεχθεί να προσομοιωθούν και τα αποτελέσματα της κάθε προσομοίωσης με γραφικό τρόπο. Τα συμπεράσματα που απορρέουν από τον έλεγχο του κάθε συστήματος και την επίδοση του αυτο--ρυθμιζόμενου PID ελεγκτή σχολιάζονται σε βάθος. Έτσι, καθίσταται εμφανές ποιες είναι οι δυνατότητες και ποιοι οι περιορισμοί του αυτο--ρυθμιζόμενου PID ελεγκτή μέσα από την εφαρμογή του σε κλασικά συστήματα αυτομάτου ελέγχου.

Στο τέλος της εργασίας παρουσιάζεται η Βιβλιογραφία. Στην ενότητα αυτή μπορεί να ανατρέξει ο αναγνώστης για να δει περισσότερες πληροφορίες για τεχνικές που αναφέρονται στο κείμενο ή για μια πιο ολοκληρωμένη και σε βάθος μαθηματική ανάλυση των μεθόδων και των συστημάτων που χρησιμοποιούνται.