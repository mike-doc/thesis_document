%!TEX root = ../main.tex



\section{Εισαγωγή}

\lettrine[findent=2pt]{\fbox{\textbf{Η}}}{} ρύθμιση του PID ελεγκτή έχει να κάνει με την απόδοση τιμών στο συντελεστή κάθε όρου, έτσι ώστε καθένας από αυτούς να επηρεάσει θετικά την απόκριση του συστήματος, ενώ ταυτόχρονα να μετριαστούν όσο γίνεται περισσότερο τα αρνητικά του κάθε όρου. Ο τύπος της διεργασίας επηρεάζει το τι είναι επιθυμητό από τον έλεγχό της. Για παράδειγμα, κάποιες διεργασίες μπορεί να μην ανέχονται υπέρβαση (\emph{overshoot}) της απόκρισής τους και αυτό να θέτει και περιορισμούς στο χρόνο ανύψωσης (\emph{rise time}). Αντιθέτως, άλλες διεργασίες μπορεί να παρουσιάζουν ανοχή σε ένα ποσοστό υπέρβασης και έτσι να επιτρέπουν να αυξηθεί ο χρόνος ανύψωσης.

Παρόλο που η υλοποίηση του PID ελεγκτή είναι σχετικά ευθύς και απλή διαδικασία, η σωστή ρύθμισή του είναι πιο περίπλοκο ζήτημα. Αυτό συμβαίνει επειδή απαιτεί κατανόηση του τρόπου με τον οποίο κάθε όρους του PID επηρεάζει τη συνολική απόκριση. Ένας κακώς ρυθμισμένος PID ελεγκτής θα εμφανίσει αρκετά προβλήματα απόδοσης όπως: ταλαντώσεις, μη επαρκή απόσβεση, υπέρβαση, αργούς χρόνους ανόδου ή καθόδου και άλλα. Σε αυτό το κεφάλαιο θα γίνει μια περιγραφή κάποιων διαδεδομένων τεχνικών ρύθμισης που συναντάει κανείς στη βιβλιογραφία (υπάρχουν πολλές πιο εξελιγμένες μέθοδοι ρύθμισης αλλά υπόκεινται σε εμπορικές πατέντες) καθώς και των προβλημάτων που οφείλονται σε κακή ρύθμιση του PID ελεγκτή.

Η ανάλυση των τεχνικών ρύθμισης θα γίνει με αναφορά στα κέρδη $K_p$, $K_i$ και $K_d$, δηλαδή θα αφορά την παράλληλη μορφή του PID ελεγκτή της Εξίσωσης (\ref{eq:parallel_pid}) και όχι την τυποποιημένη μορφή της Εξίσωσης (\ref{eq:standard_pid}), παρόλο που η δεύτερη χρησιμοποιήθηκε στην εργασία αυτή. Αυτό συμβαίνει γιατί ο αναγνώστης είναι πιο εύκολο να καταλάβει πώς επηρεάζουν τα κέρδη την απόκριση του συστήματος χρησιμοποιώντας την πρώτη εξίσωση. Φυσικά, οι ίδιοι κανόνες ισχύουν και για τις παραμέτρους $T_i$ και $T_d$ της τυποποιημένης μορφής, έχοντας πάντα υπόψιν τις εξισώσεις που συνδέουν τις παραμέτρους των δύο αυτών σχέσεων.

\section{Τεχνικές ρύθμισης}

\subsection{Χειροκίνητη ρύθμιση}

Η πρώτη και πιο φυσική μέθοδος είναι η χειροκίνητη ρύθμιση του ελεγκτή από έναν χειριστή. Σε αυτή τη μέθοδο αρχικά τα κέρδη $K_i$ και $K_d$ ισούνται με το μηδέν. Στη συνέχεια, αυξάνουμε το αναλογικό κέρδος $K_p$ μέχρι το σύστημα να αρχίσει να ταλαντώνεται ελαφρώς και ο χρόνος ανύψωσης να είναι ικανοποιητικός. Έπειτα, αυξάνουμε το ολοκληρωτικό κέρδος $K_i$ έως ότου να εξαλειφθεί το σφάλμα μόνιμης κατάστασης, σε λογικό χρόνο για τη συγκεκριμένη διεργασία. Σε αυτό το σημείο πρέπει να έχουμε στο μυαλό μας ότι μεγαλύτερο ολοκληρωτικό κέρδος σημαίνει και μεγαλύτερη αστάθεια του συστήματος. Τέλος, αν χρειάζεται, αυξάνουμε το κέρδος παραγώγου $K_d$ για να βελτιώσουμε τη σταθερότητα του συστήματος καθώς και την απόκρισή του σε αλλαγή φορτίου ή σε κάποια διαταραχή. Εδώ πρέπει να έχουμε υπόψη μας ότι μεγάλο κέρδος παραγώγου θα προκαλέσει υπερβολική ανταπόκριση σε μεταβολές και υπέρβαση.

Ένας βρόχος PID που έχει ρυθμιστεί για να επιτύχει μια γρήγορη απόκριση συνήθως θα υπερβεί ελαφρώς την επιθυμητή τιμή ως συνέπεια της γρήγορης ``ρύθμισης" του. Εντούτοις, μερικά συστήματα δεν μπορούν να ανεχθούν οποιαδήποτε υπέρβαση, οπότε στην περίπτωση αυτή απαιτείται σύστημα κλειστού βρόχου, το οποίο θα πρέπει να έχει μικρότερο αναλογικό κέρδος $K_p$ από αυτό ενός συστήματος που μπορεί να ανεχτεί κάποιο ποσοστό υπέρβασης. Στον Πίνακα \ref{table:parameters} φαίνεται πώς ο κάθε όρος του ελεγκτή επηρεάζει την απόκριση του συστήματος υπό έλεγχο.

\begin{table}[H]
\begin{center}
\begin{tabular}{ |c|c|c|c|c|c| }
\hline
\thead{Παράμετρος} & \thead{Χρόνος \\ ανύψωσης} & \thead{Υπέρβαση} & \thead{Μόνιμο \\ σφάλμα} & \thead{Χρόνος \\ ηρεμίας} & \thead{Ευστάθεια}\\ \hline
\thead{$K_p$} & \thead{Μείωση} & \thead{Αύξηση} & \thead{Μείωση} & \thead{Μικρή \\ αλλαγή} & \thead{Χειροτέρευση} \\ \hline
\thead{$K_i$} & \thead{Μείωση} & \thead{Αύξηση} & \thead{Εξάλειψη} & \thead{Αύξηση} & \thead{Χειροτέρευση} \\ \hline
\thead{$K_d$} & \thead{Μικρή \\ αλλαγή} & \thead{Μείωση} & \thead{Καμία \\ αλλαγή} & \thead{Μείωση} & \thead{Βελτίωση} \\
\hline
\end{tabular}
\caption{Επίδραση στο σύστημα, της αύξησης κάθε παραμέτρου ανεξάρτητα από τις άλλες}
\label{table:parameters}
\end{center}
\end{table}

\subsection{Μέθοδος Ziegler–Nichols}

Ίσως η πιο γνωστή μέθοδος ρύθμισης ενός PID ελεγκτή είναι η \emph{Ziegler-Nichols}. Η μέθοδος αυτή έχει πάρει το όνομά της από τους \emph{John G. Ziegler} και \emph{Nathaniel B. Nichols} που την παρουσίασαν το $1942$ και χρησιμοποιείται ακόμα και σήμερα. Όπως και στην προηγούμενη μέθοδο, στην αρχή τα κέρδη $K_i$ και $K_d$ τίθενται ίσα με το μηδέν. Στη συνέχεια αυξάνουμε το κέρδος $K_p$ μέχρι το σύστημα αυξάνεται μέχρι την τιμή που θα οδηγήσει το σύστημα να εκτελεί ταλαντώσεις σταθερού πλάτους και σταθερής περιόδου. Η τιμή αυτή του κέρδους ονομάζεται απόλυτο κέρδος, $\boldsymbol{K_u}$, και η περίοδος των ταλαντώσεων ονομάζεται απόλυτη περίοδος, $\boldsymbol{T_u}$. Αυτές τις δύο παραμέτρους τις χρησιμοποιούμε για να ορίσουμε τα κέρδη όπως φαίνεται στον Πίνακα \ref{table:zn_method}. Στον πίνακα αυτόν έχουν υπολογιστεί οι παράμετροι $T_i$ και $T_d$ της τυποποιημένης μορφής του PID αλγορίθμου \cite{ziegler-nichols}. Αυτό έγινε επειδή αυτοί οι παράμετροι θα χρησιμοποιηθούν αργότερα για την αυτόματη ρύθμιση του ελεγκτή της εργασίας.


\begin{table}[H]
 \begin{center}
 \begin{tabular}{|c|c|c|c|}
 \hline
 Τύπος ελέγχου & $K_p$ & $T_i$ & $T_d$ \\ \hline
 P & $0.50K_u$ & - & - \\ \hline
 PI & $0.45K_u$ & $T_u/1.2$ & - \\ \hline
 PD & $0.80K_u$ & - & $T_u/8$ \\ \hline
 PID & $0.60K_u$ & $T_u/2$ & $T_u/8$ \\ \hline
 \end{tabular}
 \caption{Παράμετροι Ziegler-Nichols ανάλογα τον τύπο του ελεγκτή}
 \label{table:zn_method}
 \end{center}
\end{table}

Οι κανόνες ρύθμισης Ziegler-Nichols, συνήθως οδηγούν σε συστήματα με ιδιαίτερα υψηλή υπέρβαση και ``επιθετική" (``aggressive") απόκριση. Δεν παρέχουν δηλαδή μια έτοιμη λύση ρύθμισης αλλά τις περισσότερες φορές προσφέρουν ένα αρκετά ικανοποιητικό σημείο εκκίνησης από το οποίο ο χειριστής μπορεί να ξεκινήσει να τροποποιεί τα κέρδη έτσι ώστε να έχει την επιθυμητή απόκριση.




%\section{Μοντελοποίηση προβλήματος}
%
%\lettrine[findent=2pt]{\fbox{\textbf{Π}}}{ριν} δούμε στην πράξη πώς λειτουργούν οι τεχνικές super resolution θα πρέπει να περιγράψουμε το πρόβλημα με το οποίο θα δουλέψουμε με μαθηματικούς όρους. Η μοντελοποίηση αυτή θα μας επιτρέψει να χρησιμοποιήσουμε μαθηματικά εργαλεία και τεχνικές και να ορίσουμε σε μια αυστηρή ``γλώσσα'' (αυτή των μαθηματικών) τις λειτουργίες που επιτελούνται από κάθε μέθοδο, ούτως ώστε να πετύχουμε το τελικό αποτέλεσμα.
%
%Ξεκινώντας, θα πρέπει να περιγράψουμε τον τρόπο με τον οποίο λαμβάνουμε εικόνες χαμηλής ανάλυσης από μια φυσική σκηνή μέσω μιας κάμερας. Η κάμερα σαν όργανο καταγραφής εισάγει κάποιες ατέλειες, όπως είδαμε στο προηγούμενο κεφάλαιο. Τέτοιες ατέλειες μπορεί να είναι σφάλματα καταγραφής από τον αισθητήρα της κάμερας, θόλωμα λόγω αστοχιών των οπτικών στοιχείων κλπ. Επιπλέον, αν λάβουμε διαδοχικές εικόνες μιας φυσικής σκηνής, οι εικόνες αυτές περιμένουμε να έχουν κάποιες μετατοπίσεις ως προς αυτό που απεικονίζουν. Οι μετατοπίσεις αυτές μπορεί να οφείλονται είτε στον άνθρώπινο παράγοντα (που χειρίζεται την κάμερα) είτε στο αντικείμενο της φυσικής σκηνής που μπορεί να μην είναι σταθερό. Στην εικόνα που λαμβάνεται τελικά, λαμβάνονται δείγματα σε χαμηλή χωρική συχνότητα και λόγω ατελειών του οργάνου μπορεί να έχουμε και παρουσία θορύβου. Για μια εικόνα υψηλής ανάλυσης $\bm{X}$ (την οποία θα προσπαθήσουμε να ανακατασκευάσουμε), μπορούμε να αναπτύξουμε το παραπάνω μοντέλο ως εξής:
%\begin{equation} \label{eq:hrmodel}
%\bm{Y}_i=\bm{S}_i\bm{T}_i\bm{H}_i\bm{X}+\bm{n}_i
%\end{equation}
%όπου ορίζουμε για την i-οστή εικόνα χαμηλής ανάλυσης $\bm{Y}_i$ τους τελεστές που ενεργούν στην $\bm{X}$ : 
%\begin{itemize}
%	\item $\bm{S}_{i}$ για υποδειγματοληψία, 
%	\item $\bm{T}_{i}$ για μετατόπιση, 
%	\item $\bm{H}_{i}$ για θόλωμα (blurring), 
%	\item $\bm{n_{i}}$ για προσθετικό θόρυβο.
%\end{itemize}
%Οι υποθέσεις που κάνουμε για το παραπάνω πρόβλημα είναι ότι το blurring είναι ίδιο σε όλο το χώρο και είναι γνωστό στον αλγόριθμο super resolution, ο θόρυβος είναι λευκός Gaussian με την ίδια διασπορά σε όλες τις εικόνες χαμηλής ανάλυσης και ότι ο γεωμετρικός μετασχηματισμός αφορά μόνο την καθολική μετατόπιση. 
%
% Στην περίπτωση που θεωρήσουμε αμελητέο θόρυβο και θόλωμα, τα παραπάνω βήματα αρκούν για να υπολογίσουμε την εικόνα υψηλής ανάλυσης $\bm{X}$, την οποία λαμβάνουμε σαν αποτέλεσμα του αλγορίθμου.
%
%\begin{algorithm}
%
% \KwData{$\bm{dx}$, $\bm{dy}$, $\bm{Y}_i$, $N$, $W$, $\bm{H}$, $S$}
% \KwOut{High resolution reconstructed $\bm{X}$}
% $\bm{X} \gets 0$\; 
% \For{$n \gets 1$ \textbf{to} $N$}{
% 	$\bm{\tilde{Y}} \gets \bm{Y}_i$\;
%    $\bm{i} \gets 1:W/S$\;
%    $\bm{j} \gets 1:H/S$\;
%    $\bm{px}=\bm{i}*S+\bm{dx}_n$\;
%    $\bm{py}=\bm{j}*S+\bm{dy}_n$\;
%    $\bm{X}_{px,py}=\bm{\tilde{Y}}_{i,j}$\;
%    }
%    %\vspace{-1.5em}
% \caption{Ανακατασκευή shift-add fusion}\label{algo:sr_fusion}
%\end{algorithm}
%
%Συγκρίνοντας τη μέθοδο αυτή με την ανακατασκευή μέσω του μετασχηματισμού Fourier, παρατηρούμε την απλότητα και την αποδοτικότητά της καθώς βρίσκει την εικόνα υψηλής ανάλυσης μόνο με μετακινήσεις pixel. 