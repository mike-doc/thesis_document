\pagestyle{plain}
\begin{center}
{\LARGE Περίληψη}\\[1cm]
\end{center}

%\lettrine[findent=2pt]{\fbox{\textbf{H}}}{ } εργασία αυτή ασχολείται με ένα ιδιαίτερα ενδιαφέρον ζήτημα στο χώρο της επεξεργασίας σημάτων και εικόνων, την ανάλυση (\emph{resolution}). Παρόλο που σήμερα στον κόσμο μας έχουμε καταφέρει να δημιουργήσουμε συσκευές με μεγάλη ευαισθησία καταγραφής, μεγάλο χώρο αποθήκευσης καθώς και υψηλούς ρυθμούς μετάδοσης και επεξεργασίας δεδομένων, εντούτοις υπάρχουν εφαρμογές όπου η φύση τους είναι τέτοια που δε μας επιτρέπει να επωφεληθούμε σε μεγάλο βαθμό από την πρόοδο που έχει σημειωθεί. Μια τέτοια εφαρμογή είναι οι θερμικές εικόνες και θα δούμε στη συνέχεια της εργασίας τους λόγους εκείνους που την καθιστούν ``ιδιαίτερη".

\lettrine[findent=2pt]{\fbox{\textbf{Η}}}{ } εργασία αυτή ασχολείται με την αυτόματη ρύθμιση (\emph{self--regulation}) PID ελεγκτών. Παρόλο που οι PID ελεγκτές αποτελούν ένα πολύ διαδεδομένο είδος ελεγκτών με ευρεία χρήση σε βιομηχανικές, και όχι μόνο, εφαρμογές η σωστή ρύθμιση τους απαιτεί εμπειρία από το χειριστή και συνήθως αποτελεί χρονοβόρα διαδικασία. Μέσω προσομοίωσης στο περιβάλλον LabVIEW, γίνεται προσπάθεια να αυτοματοποιηθεί η διαδικασία αυτή και να φανεί ποιες είναι οι δυνατότητες και ποιοι οι περιορισμοί ενός αυτο--ρυθμιζόμενου PID ελεγκτή μέσα από την εφαρμογή του σε κλασικά συστήματα αυτομάτου ελέγχου.