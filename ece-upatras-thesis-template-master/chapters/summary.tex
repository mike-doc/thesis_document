\pagestyle{plain}
\begin{center}
{\LARGE Περίληψη}\\[1cm]
\end{center}

%\lettrine[findent=2pt]{\fbox{\textbf{H}}}{ } εργασία αυτή ασχολείται με ένα ιδιαίτερα ενδιαφέρον ζήτημα στο χώρο της επεξεργασίας σημάτων και εικόνων, την ανάλυση (\emph{resolution}). Παρόλο που σήμερα στον κόσμο μας έχουμε καταφέρει να δημιουργήσουμε συσκευές με μεγάλη ευαισθησία καταγραφής, μεγάλο χώρο αποθήκευσης καθώς και υψηλούς ρυθμούς μετάδοσης και επεξεργασίας δεδομένων, εντούτοις υπάρχουν εφαρμογές όπου η φύση τους είναι τέτοια που δε μας επιτρέπει να επωφεληθούμε σε μεγάλο βαθμό από την πρόοδο που έχει σημειωθεί. Μια τέτοια εφαρμογή είναι οι θερμικές εικόνες και θα δούμε στη συνέχεια της εργασίας τους λόγους εκείνους που την καθιστούν ``ιδιαίτερη".

\lettrine[findent=2pt]{\fbox{\textbf{Η}}}{ } εργασία αυτή ασχολείται με τη δημιουργία ενός αυτο--ρυθμιζόμενου PID ελεγκτή. Παρόλο που οι PID ελεγκτές αποτελούν ένα πολύ διαδεδομένο είδος ελεγκτών με ευρεία χρήση σε βιομηχανικές, και όχι μόνο, εφαρμογές η σωστή ρύθμιση τους απαιτεί εμπειρία από το χειριστή και συνήθως αποτελεί χρονοβόρα διαδικασία.

Πολλοί τρόποι έχουν αναπτυχθεί για την αυτόματη ρύθμιση των κερδών ενός PID ελεγκτή έτσι ώστε ο έλεγχος που προσφέρει να πληρεί κάποιες απαιτήσεις. Στην εργασία αυτή θα ασχοληθούμε τη μέθοδο \emph{Relay} πάνω στην οποία βασίστηκε ο συγκεκριμένος αυτο--ρυθμιζόμενος PID ελεγκτής που σχεδιάστηκε. Ο κώδικας καθώς και οι προσομοιώσεις γίνονται σε περιβάλλον LabVIEW.

Αρχικά, γίνεται μια αναφορά στο περιβάλλον LabVIEW, καθώς δεν αποτελεί μια κλασική, δομική γλώσσα προγραμματισμού. Το επόμενο κεφάλαιο ασχολείται με τους PID ελεγκτές, την ιστορική τους εξέλιξη καθώς και τις αρχές που διέπουν τη λειτουργία τους. Στη συνέχεια, αναλύονται με λεπτομέρεια οι τεχνικές ρύθμισης που χρησιμοποιήθηκαν για την υλοποίηση του συγκεκριμένου ελεγκτή. Επίσης, αναλύονται και τα προβλήματα και οι περιορισμοί που καθιστούν τη ρύθμιση, ενός απλού ελεγκτή, δύσκολη υπόθεση. 

Στο τέταρτο κεφάλαιο, παρουσιάζεται η θεωρία της ``Relay" μεθόδου που χρησιμοποιείται για την αυτόματη ρύθμιση του ελεγκτή καθώς και το πρόγραμμα που αποτελεί τον PID ελεγκτή. Γίνεται εκτενής χρήση εικόνων, τόσο του εποπτικού πάνελ όσο και του δομικού διαγράμματος κάθε συνάρτησης που υλοποιεί διαφορετική λειτουργία. Επίσης, σχολιάζεται η λογική πίσω από κάθε σχεδιαστική επιλογή και οι λόγοι που οδήγησαν σε αυτή.

Τέλος, στο πέμπτο κεφάλαιο, παρουσιάζονται τα συστήματα που έχουν επιλεχθεί να προσομοιωθούν και τα αποτελέσματα της κάθε προσομοίωσης με γραφικό τρόπο. Τα συμπεράσματα που απορρέουν από τον έλεγχο του κάθε συστήματος και την επίδοση του αυτο--ρυθμιζόμενου PID ελεγκτή σχολιάζονται σε βάθος. Έτσι, καθίσταται εμφανές ποιες είναι οι δυνατότητες και ποιοι οι περιορισμοί ενός αυτο--ρυθμιζόμενου PID ελεγκτή μέσα από την εφαρμογή του σε κλασικά συστήματα αυτομάτου ελέγχου.